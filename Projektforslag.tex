
Projektforslag til software/datalogi

Når man på store undervisningsinstitutioner skal skemalægge undervisningen, er der altid et stort bøvel. Alles interesser skal gerne tilses, og der er næsten aldrig nok plads. Hvis der i løbet af et kursus opstår sygdom, hvordan ændrer man så skemaet for at indhente tabt tid?

målet er at opstille, undersøge og implementere en model, der kan assistere med skemalægning på undervisningensinstitutioner. Der skal stages højde for begrænsede resourcer, både som lokaler og kursusholderer. Der bør være overvejelser omkring sygedage og tid imellem forelæsninger til at lave lektier.

Datalogiske problemstillinger indebærer graf-teori og tildels spilteori.

Dette projekt vil give indsigt i P versus NP problemstillingen, hvilket er en stor mundfuld, men et meget relevant emne for software/datalogi. 

forslaget er udarbejdet af SW1-B2-6, 2013.
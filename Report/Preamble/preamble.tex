%!TEX root = ../Master.tex
\documentclass[a4paper,12pt,twoside,openright]{memoir} % Brug openright hvis chapters skal starte på højresider; openany, oneside

%%%% PACKAGES %%%%

\usepackage[T1]{fontenc}
\usepackage[ansinew]{inputenc}
\usepackage{fourier}
\usepackage[english]{babel}
\usepackage{csquotes}
\usepackage[table]{xcolor}
\usepackage{microtype}			% Justering af elementer

%\pretolerance=2500 											% Gør det muligt at justre afstanden mellem ord (højt tal, mindre orddeling og mere space mellem ord)
%\usepackage{ulem}                       % Gennemstregning af ord med koden \sout{}
																			
% ¤¤ Figurer og tabeller – floats  ¤¤ %
%\usepackage{flafter}										% Sørger for at dine floats ikke optræder i teksten før de er sat ind.
%\usepackage{multirow}                		% Fletning af rækker
%\usepackage{hhline}                   	% Dobbelte horisontale linier
%\usepackage{multicol}         	        % Fletning af kolonner
%\usepackage{colortbl} 									% Muligøre farver i tabeller
%\usepackage{here}												% Gør det muligt at placere figurer hvor du vil.   \begin{figure}[!h] % Will not be floating.
%\usepackage{wrapfig}										% Indsættelse af figurer omsvøbt af tekst. \begin{wrapfigure}{Placering}{Størrelse}
%\usepackage{floatflt}										% Indsættelse af tabeller omsvøbt af tekst.
%\pdfoptionpdfminorversion=6							% Muliggør inkludering af pdf dokumenter, af version 1.6 og højere
%\usepackage{placeins}	
% ¤¤ Matematiske formler og maskinkode ¤¤
\usepackage{amsmath} 	% Bedre matematik og ekstra fonte
%\usepackage{textcomp}                 	% Adgang til tekstsymboler
%\usepackage{mathtools}									% Udvidelse af amsmath-pakken.
%\usepackage{eso-pic}										% Tilføj billedekommandoer på hver side
%\usepackage{lipsum}											% Dummy text \lipsum[..]

% ¤¤ PDF og billede optimering ¤¤ %
\usepackage{graphicx} 									% Pakke til jpeg/png billeder
\graphicspath{{Images/}}
\usepackage{tikz}
\usepackage{subcaption}

 % support for sidste side page number
%\usepackage{lastpage}
\usepackage{totpages}
\usepackage{intcalc}
\usepackage{refcount}


% ¤¤ Referencer, bibtex og url'er ¤¤ %
\usepackage{url}

%% Define a new 'leo' style for the package that will use a smaller font.
\makeatletter
\def\url@leostyle{%
  \@ifundefined{selectfont}{\def\UrlFont{\sf}}{\def\UrlFont{\small\ttfamily}}}
\makeatother
%% Now actually use the newly defined style.
\urlstyle{leo}

\usepackage[english]{varioref}
\usepackage[colorlinks]{hyperref} 
\usepackage{cleveref}						% Giver flere bedre mulighed for at lave krydshenvisninger
%\usepackage{natbib}											% Litteraturliste med forfatter-år og nummererede referencer


\usepackage[marginclue,inline,draft,mode=multiuser,silent]{fixme}
\FXRegisterAuthor{si}{asi}{Simon}
\FXRegisterAuthor{jo}{ajo}{Jonas}
\FXRegisterAuthor{ka}{aka}{Kasper}
\FXRegisterAuthor{an}{aan}{Anders}
\FXRegisterAuthor{ca}{aca}{Carsten}
\FXRegisterAuthor{al}{aal}{Alexander}
\FXRegisterAuthor{ch}{ach}{Christian}


% Indsæt rettelser og lignende med \fixme{...} Med final i stedet for draft, udløses en error for hver fixme, der ikke er slettet, når rapporten bygges.  

\newcommand{\appendixpages}[0]{
     \intcalcAdd
      {1}
      {
        \intcalcSub
        {\getpagerefnumber{TotPages}}
        {\getpagerefnumber{appendix_start}}
        }
    }

%%%% CUSTOM SETTINGS %%%%

% ¤¤ Marginer ¤¤ %
\setlrmarginsandblock{3.5cm}{2.5cm}{*}	% \setlrmarginsandblock{Indbinding}{Kant}{Ratio}
\setulmarginsandblock{2.5cm}{3.0cm}{*}	% \setulmarginsandblock{Top}{Bund}{Ratio}
\checkandfixthelayout 									% Laver forskellige beregninger og sætter de almindelige længder op til brug ikke memoir pakker

%	¤¤ Afsnitsformatering ¤¤ %
\setlength{\parindent}{0mm}           	% Størrelse af indryk
\setlength{\parskip}{3mm}          			% Afstand mellem afsnit ved brug af double Enter
\linespread{1,2}												% Linie afstand



\usepackage[
    bibencoding=utf8,
    hyperref=true,
    backend=biber
]{biblatex}
\bibliography{Bibtex/References.bib}

% ¤¤ Indholdsfortegnelse ¤¤ %
\setsecnumdepth{subsection}		 			% Dybden af nummerede overkrifter (part/chapter/section/subsection)
%\maxsecnumdepth{subsubsection}					% Ændring af dokumentklassens grænse for nummereringsdybde
\settocdepth{subsection} 								% Dybden af indholdsfortegnelsen

% ¤¤ Visuelle referencer ¤¤ %
		 	% Giver mulighed for at ens referencer bliver til klikbare hyperlinks. .. [colorlinks]{..}
\hypersetup{pdfborder = 0}							% Fjerner ramme omkring links i fx indholsfotegnelsen og ved kildehenvisninger ¤¤
\hypersetup{														%	Opsætning af hyperlinks
    colorlinks = false,
    linkcolor = black,
    anchorcolor = black,
    citecolor = black
}


% ¤¤ Opsætning af figur- og tabeltekst ¤¤ %
 % 	\captionnamefont{
 % 		\small\bfseries\itshape}						% Opsætning af tekstdelen ("Figur" eller "Tabel")
 %  \captiontitlefont{\small}							% Opsætning af nummerering
 %  \captiondelim{. }											% Seperator mellem nummerering og figurtekst
 %  \hangcaption													%	Venstrejusterer flere-liniers figurtekst under hinanden
 %  \captionwidth{\linewidth}							% Bredden af figurteksten
	% %\setlength{\belowcaptionskip}{10pt}		% Afstand under figurteksten

% ¤¤ Sidehoved ¤¤ %
\pagestyle{plain}
		

%Appendix opsætning. Toc er sætter appendix kapitlerne i indholdfortegnelsen og head sætter en overskrift til appendix kapitlet

%\usepackage{appendix}
%\usepackage[toc,page]{appendix}


% ¤¤ Kapiteludssende ¤¤ %

\definecolor{numbercolor}{gray}{0.7}
\newif\ifchapternonum

\makechapterstyle{jenor}{
  \renewcommand\printchaptername{}
  \renewcommand\printchapternum{}
  \renewcommand\printchapternonum{\chapternonumtrue}
  \renewcommand\chaptitlefont{\fontfamily{pbk}\fontseries{db}\fontshape{n}\fontsize{25}{35}\selectfont\raggedleft}
  \renewcommand\chapnumfont{\fontfamily{pbk}\fontseries{m}\fontshape{n}\fontsize{1in}{0in}\selectfont\color{numbercolor}}
  \renewcommand\printchaptertitle[1]{%
    \noindent
    \ifchapternonum
    \begin{tabularx}{\textwidth}{X}
    {\let\\\newline\chaptitlefont ##1\par}
    \end{tabularx}
    \par\vskip-2.5mm\hrule
    \else
    \begin{tabularx}{\textwidth}{Xl}
    {\parbox[b]{\linewidth}{\chaptitlefont ##1}} & \raisebox{-15pt}{\chapnumfont \thechapter}
    \end{tabularx}
    \par\vskip2mm\hrule
    \fi
  }
}

%\chapterstyle{jenor}
\chapterstyle{demo}

% ¤¤ Fjerner den vertikale afstand mellem listeopstillinger og punktopstillinger ¤¤ %
%\let\olditemize=\itemize							
%\def\itemize{\olditemize\setlength{\itemsep}{-1ex}}
%\let\oldenumerate=\enumerate						
%\def\enumerate{\oldenumerate\setlength{\itemsep}{-1ex}}


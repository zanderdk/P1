%!TEX root = ../../Master.tex
\section{Graph theory}

Graph theory is frequently used in computer science to model some kind of relationship between objects. These objects could be anything. Graph theory is a preferred method to model building complexes, because it can precisely model how e.g. a hallway in a hospital is connected to a room.

In graph theory objects are called \enquote{nodes} or \enquote{vertices}. The two terms can be used interchangeably. In this report, vertices will describe locations or POI (points of interest) in a hospital. Another term in graph theory is an edge. This is basically connecting two vertices and thereby providing a relationship between these vertices. An edge can be seen as a possible route connecting one location to another. Now in order to describe the relationship between two vertices, we use an weighted graph in which an edge has a number attribute called a weight. A weight can describe the time, distance or any other metric that in some way can describe how two connected vertices are related \cite{wiki_graph_glos,MIT2012}. 

For an example of a weighted graph, see \cref{fig:graph}. All letters mark vertex ids. So circles are vertices. Edges are represented by lines connecting circles. The weight of the edge is represented by the numbers next to the line.

We can describe these definitions formally.\cite{MIT2012}
\begin{mydef}
	A graph $G$ is a pair of sets $(V,E)$ where $V$ is a non-empty set of items called vertices or nodes. $E$ is a set of 2-item subsets of $V$ called edges.
\end{mydef}

  \begin{figure}[ht!]
    \centering
    \includegraphics[width=0.5\textwidth]{eksempel_graf.eps}
    \caption{An example of a weighted graph}
    \label{fig:graph}
  \end{figure}

% \begin{figure}[ht!]
%     \centering
%     \includegraphics[width=0.5\textwidth]{6n-graf.eps}
%     \caption{A labeled simple graph with vertex set $V = \left\{ {1, 2, 3, 4, 5, 6} \right\} $ and edge set $E = \left\{ \left\{ {1,2}\right\}, \left\{ {1,5}\right\}, \left\{ {2,3}\right\}, \left\{ {2,5}\right\}, \left\{ {3,4}\right\}, \left\{ {4,5} \right\} , \left\{ {4,6} \right\} \right\}$. \cite{wiki_graph_glos}}
%     \label{fig:labeled_graph}
%   \end{figure}

\section{Directed and Undirected Graphs}
\annote{refference ti l figuren virker ikke}
Figure \cref{labeled_Directed_undirected} shows two graphs, one which is directed and another one which is undirected. Graph 1 is undirected and don't have a specific direction. Graph 2 is directed which means the edges is one way only. The arrows indicates the direction the edges allow. This means vertex 1 can go to vertex 2 and 3, but neither can go back. 


\begin{figure}[ht!]
    \centering
    \includegraphics[width=0.5\textwidth]{Directed_undirected.png}
    \label{fig:labeled_Directed_undirected}
    \caption{Graph 1 is directed were graph 2 is undirected \newline Picture from Differencebetween.com \cite{dir_pic}}
  \end{figure}

\subsection{Representation of graphs}

Two standard ways to represent a graph exists: as a collection of adjacency lists or as an adjacency matrix. Both representations support directed and undirected graphs. Each representation method has its merits \cite{Cormen2009}.

\subsubsection{Adjacency list}
Adjacency lists provide a very compact way to represent \textbf{sparse} graphs. The keyword is sparse. Adjacency lists are only compact and the preferred representation of graphs if the number of edges in a graph is much less than the number of vertices squared. \Cref{fig:graph,fig:labeled_Directed_undirected} are represented as a collection of adjacency lists.

\subsubsection{Adjacency matrix}
Adjacency matrices are usually the representation of choice when the graph is \textbf{dense}. This means that adjacency matrices should be used when the number of edges is close to the number of vertices squared. Adjacency matrices also has the merit of providing quick lookup of which vertices are connected. See \cref{fig:adjacency_matrix} for an example. Row and column numbers are vertex ids. The contents of the matrix represents the weights in the graph. So when finding the weight connecting vertex 3 and 4, one would look in row 4 column 3. The weight would then be 1. If the weight is 0, the two vertices are not connected.

\begin{figure}[ht!]
    \centering
    \includegraphics[width=0.5\textwidth]{adjacency_matrix}
    \caption{An example of an adjacency matrix \cite{Cormen2009}.}
    \label{fig:adjacency_matrix}
  \end{figure}



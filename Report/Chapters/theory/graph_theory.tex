%!TEX root = ../../Master.tex
\section{Graph theory}

Graph theory is frequently used in computer science to model some kind of relationship between objects. These objects could be anything. Graph theory is a preferred method to model building complexes, because it can precisely model how e.g. a hallway in a hospital is connected to a room.

In graph theory objects are called \enquote{nodes} or \enquote{vertices}. The two terms can be used interchangeably. In this report, vertices will describe coordinates or POI (points of interest) in a hospital. Another term in graph theory is an edge. This is basically connecting two vertices and thereby providing a relationship between these vertices. See \cref{fig:labeled_graph}. An edge can be seen as a possible route connecting one coordinate to another. Now in order to describe the relationship between two vertices, we use an weighted graph in which an edge has a number attribute called a weight. A weight can describe the time, distance or any other metric that in some way can describe how two connected vertices are related\cite{wiki_graph_glos,MIT2012}.

We can describe these definitions formally.\cite{MIT2012}
\begin{mydef}
	A graph $G$ is a pair of sets $(V,E)$ where $V$ is a non-empty set of items called vertices or nodes. $E$ is a set of 2-item subsets of $V$ called edges.
\end{mydef}

\begin{figure}[ht!]
    \centering
    \includegraphics[width=0.5\textwidth]{6n-graf.eps}
    \caption{A labeled simple graph with vertex set $V = \left\{ {1, 2, 3, 4, 5, 6} \right\} $ and edge set $E = \left\{ \left\{ {1,2}\right\}, \left\{ {1,5}\right\}, \left\{ {2,3}\right\}, \left\{ {2,5}\right\}, \left\{ {3,4}\right\}, \left\{ {4,5} \right\} , \left\{ {4,6} \right\} \right\}$. \cite{wiki_graph_glos}}
    \label{fig:labeled_graph}
  \end{figure}

\section{Directed and Undirected Graphs}
\annote{refference ti l figuren virker ikke}
Figure \cref{labeled_Directed_undirected} shows two graphs, one which is directed and another one which is undirected. Graph 1 is undirected and don't have a specific direction. Graph 2 is directed which means the edges is one way only. The arrows indicates the direction the edges allow. This means vertex 1 can go to vertex 2 and 3, but neither can go back. 


\begin{figure}[ht!]
    \centering
    \includegraphics[width=0.5\textwidth]{Directed_undirected.png}
    \label{fig:labeled_Directed_undirected}
    \caption{Graph 1 is directed were graph 2 is undirected \newline Picture from Differencebetween.com \cite{dir_pic}}
  \end{figure}
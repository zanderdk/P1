%!TEX root = ../../Master.tex
\section{Specification} % (fold)
\label{sec:specification}

What follows is a list of requirements that needs to be met, before a solution can be considered complete. These specifications are based on discoveries made in the analysis of the initial problem. Each requirement will be followed by a short description of why it is necessary for a complete solution, and references to which parts of the analysis that supports it.

\subsection{Usage Requirements}

Usage requirements are needs that must be met, before the solution can be considered to be meeting any and all anticipated users basic expectations for the interaction with the solution. This does not include any underlying computations, but only the ways of giving input to and representing of those computations, in compliance with the needs of the using stakeholders described in \cref{sec:interusers}.

\begin{itemize}
	\item The user must have the ability to personalize the generated route through the following parameters.
			\begin{itemize}
				\item Are stairs to be avoided or preferred?
				\item Are elevators to be avoided or preferred?
				\item Are disabled-friendly routes a must?
				\item Should the route be fastest in time, shortest in distance, or the least physical strenuous?
			\end{itemize}
\end{itemize}
In order to optimize navigation, the limits of systems in current use must be considered. See \cref{sec:anal_nav}. A system that does not take personalized input from a user, will not be able to consider the needs for that particular user. As described in \cref{sec:interusers} these needs can differs immensely across usergroups, and therefore it is important to be able to personalize the generated route.
The first three parameters are able to establish most boundaries the users might have, whether it is physical or psychological, while still keeping input to a minimum, another requirement we expand on later. The last parameter is less essential, since all usergroups could always use the least physical strenuous, however in order to optimize the route specifically for the user, this option should be given as well. This last parameter could also be an indirect input, based on other input, such as the travelling speed of the user and previous choices. In any way, the program should be able to choose different criterias for the choice of optimal route.

\begin{itemize}
	\item The user interface must be intuitive, simple and still give a quick overview of data representation.
\end{itemize}
It is essential to display all the nessesary information, and nothing more. However, if the information is displayed staticly, this can prove to be impossible, since the criterias for 'nessesary information' changes with the user. See both \cref{sec:anal_nav} and \cref{sec:interusers} The advantage of a software solution in this regard, is the ability to shift the information displayed, based on the users current situation. In order to fully optimize, this must be incorporated into a solution.

\begin{itemize}
	\item The maximum waiting time during route generating should not under normal conditions exceed 2 seconds.
\end{itemize}
Studies have shown that the patience of a user, is fairly low. See \cref{sec:interusers}. The user must experience that the solution better than the alternatives, before agreeing with that statement. it therefore become important to relay information before the user feels that time is being used ineffeciently.

\begin{itemize}
	\item The program must represent the route instructions either as text, audio or pictographic.
\end{itemize}
Since the solution will be dealing with a wide varity of users (See \cref{sec:interusers}), the means of relaying information to the user, should cater to the different needs of the users.


\subsection{Technical requirements}

Technical requirements are all the requirements that will only indirectly affect the user. This concerns matters on resolving navigation problems while under certain limitations, such as being indoor or momentary isolation. These requirements are the backbone of the solution, all are thus paramount for the uttermost basic functions of a solution.

\begin{itemize}
	\item The program must be able to generate the optimal route to a given destination.
\end{itemize}
Problems with current navigation systems, include relying on the users ability to figure out the optimal route with sometimes confusing information displayed by a stationaire map. See \cref{sub:map}. In a software solution this could, and should, be handled by the system.

\begin{itemize}
	\item The program must be able to position the user through the use of radio triangulation or location fingerprinting.
\end{itemize}
The ability to position the user, is the first step required in order to establish a route to any desired location.
\kanote{Er ikke helt sikker på, hvor præcist i rapporten vi underbygger hvorfor radio er godt? Kan/skal vi stryge det sidste?}

\begin{itemize}
	\item The user's movements must be traced even without communicating with external factors.
\end{itemize}
Because of the indoor environment present at a hospital, the user might momentarily loose a connection to a given positioning system. Therefore any device in charge of positioning the user, should be able to do this to a certain extend, without communcating with external factors.
\kanote{hvor skriver vi om det her?}

\begin{itemize}
	\item The program must be able to detect deviation from the route, and recalculate accordingly.
\end{itemize}	
Ability to react to deviation from the route, is one of the strongpoints of using software for navigation. A typical reason for failure while navigating, is becoming lost. Many times this realization only comes, after retracing your way back to the original route has become too complicated, or at a point, where the original route is no longer optimal. Therefore an optimal solution should have the ability to reasses the situation in case of devation, wether it was a misstep done by the user, or a circumstantial problem with the route.
\kanote{hvor skriver vi om det her?}

\begin{itemize}
	\item The program must be able to support different building complexes.
\end{itemize}
Any kind of tool takes a bit of effort to understand at first, and this solution will be no different. If the solution however was to be applied in multiple places, chances are that the user might already have had dealings with it. This could reducing the users need to familiarize with the solution, and will create a more pleasent user experience. See \cref{sec:interusers}

% section specification (end)
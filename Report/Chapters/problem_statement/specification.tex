%!TEX root = ../../Master.tex
\section{Specification} % (fold)
\label{sec:specification}

What follows is a list of requirements that needs to be met, before a solution can be considered complete. These specifications are based on discoveries made in the analysis of the initial problem. Each requirement will be followed by a short description of why it is nessesary for a complete solution, and references to which parts of the analysis that supports it.

\subsection{Usage requirements}

Usage requirements are needs that must be met, before the solution can be considered to be meeting any and all anticipated users basic expectations for the interaction with the solution. This does not include any underlying computations, but only the ways of giving input to and representing of those computations, in compliance with the needs of the using stakeholders described in \cref{sec:interusers}.

\begin{description}
	\item[Hygge hejsa, simon har briller] The user must have the ability to personalize the generated route through the following parameters.

			\begin{itemize}
				\item Are stairs to be avoided or preferred?
				\item Are elevators to be avoided or preferred?
				\item Are disabled-friendly routes a must?
				\item Should the route be fastest in time, shortest in distance, or have the least number of climbs like stairs?
			\end{itemize}

	\item The user interface must be intuitive, simple but still give a quick overview of data representation.
	\item The maximum waiting time during initial route generation must not exceed 2 seconds.
	\item The program must represent the route instructions either as text, audio or pictographic.
\end{description}


\subsection{Technical requirements}

\begin{itemize}
	\item The program must be able to generate a route to a given destination.
	\item The program must be able to position itself by the use of radio triangulation or location fingerprinting.
	\item The program must maintain the user's position without communicating with external factors.
	\item The program must be able to detect deviation from the route, and recalculate accordingly.
	\item The program must be able to support different building complexes.
\end{itemize}



\sinote{til analysen: Kilden vurderer den samlede accepterede ventetid til mellem 2 og 4 sekunder [http://cba.unl.edu/research/articles/548/download.pdf]}

\sinote{til løsningen: Vi afsætter 500 ms til selve rutegenereringen, som måles fra modtagelse af positionsdata og ruteindstillinger, til udregningen af ruten er færdiggjort. Dette giver minimum 1500 ms til at repræsentere instrukser.}

% section specification (end)
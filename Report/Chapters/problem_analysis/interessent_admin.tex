Hospitals have many different kinds of users, in this section the focus is ''the administrative group''; namely ''the regular staff'', ''the board of directors'' and ''The region''. All three groups have different traits and impacts on the hospital. \\ The focus in this section is what impact indoor navigation has on the general effectiveness of the hospital and the three groups of administrative staff.\\
Hospitals can involve large and complex buildings and welcomes numerous people every day. Some of those people are patients and others are the loved ones of those patients come to visit, some of whom may be entering for the first time. People who are not used to a hospital can become stressed or confused as they attempt to find an appointment, the cafeteria or the waiting rooms. As a consequence of this they ask the administrative staff for help.
\kanote{Er der en grund til at bruge '' istedte for "?, og administrative group er ikke 'users'}

\textbf{Regular staff}: This group can be split into several subgroups containing for example the receptionist and the ''administrative assistant''. Members of the administrative staff are often the first employees visitors will meet in a hospital, and these employees can be split into two groups in terms of how well they know the building; the new employees and the veteran employees. A new staff member may have a hard time navigating visitors or patients around in a complex facility due to a lack of knowledge of the area, whereas the veteran staff should be more knowledgeable of the facility. Even so it can still be difficult to explain a route through the facility to a visitor. Explaining it is time consuming and it will interrupt what else they might have been working on which in turn may be stressful for the staff. If people visiting the hospital are to rely on the staff for navigation it also possesses a bottleneck as each employee can only communicate with one person at a time.

%Er der en grund til at bruge \textbf{} isetdet for \paragraph{} ?
\textbf{Board of Directors and The region}: This group can be split into several subgroups containing for instance, ''economy'' and ''scheduling and quality'', normal these groups, don't meed up with the visitors directly, but these groups are Indirectly affected by the visitors and the staff. if staff are on a tight schedule, and they run into an unforeseen workload, then their schedule will slip. this can effect appointments the staff may have. The staff will then have to work overtime, and they will then have to be paid more, effecting the economy in the hospital. A ineffective hospital will cost more money for the region, and the work will / may be of lower quality, this can prevent that the money could be used for something more useful.

In the following we will cover which procedures are needed in order to realize a software solution for our initiating problem (\cref{sub:init}).

In a decision process the following phases will typically happen. \cite{Sjaelland}


\begin{itemize}
  \setlength{\itemsep}{1pt}
  \setlength{\parskip}{0pt}
  \setlength{\parsep}{0pt}
	\item \textbf{Idea and initiative phase} A problem is defined by the citizens, media or political organisation.
	\item \textbf{Preparing phase} The problem will be reviewed according to the health legislation.
	\item \textbf{Decision phase} The problem is presented and a committee is established. This committee will typically work with consultants to specify the requirements of the solution.
	\item \textbf{Implementation phase} The solution will be implemented. 
\end{itemize}


Between the decision phase and the implementation phase, the project will be announced as a public supply contract for suppliers to tender. \cite{Union2004}. Suppliers will then submit their solutions. The solution that fits the problem's requirements best, gets the contract. 

Hospitals have many different kinds of users, in this section the focus is ''the administrative group''; namely ''the regular staff'', ''the board of directors'' and ''The region''. All three groups have different traits and impacts on the hospital. \\ The focus in this section is what impact indoor navigation has on the general effectivity of the hospital and the three groups of administrative staff.\\
Hospitals can involve large and complex buildings and welcomes numerous people every day. Some of those people are patients and others are the loved ones of those patients come to visit, some of whom may be entering for the first time. People who are not used to a hospital can become stressed or confused as they attempt to find an appointment, the cafeteria or the waiting rooms. As a consequence of this, they ask the administrative staff for help.

\textbf{Regular staff}: This group can be split into several subgroups containing for example the receptionist and the ''administrative assistant''. Members of the administrative staff are often the first employees visitors will meet in a hospital, and these employees can be split into two groups in terms of how well they know the building; the new employees and the veteran employees. A new staff member may have a hard time navigating visitors or patients around in a complex facility due to a lack of knowledge of the area, whereas the veteran staff should be more knowledgeable of the facility. Even so it can still be difficult to explain a route through the facility to a visitor. Explaning it is time consuming and it will interrupt what else they might have been working on which in turn may be stressful for the staff. If people visiting the hospital are to rely on the staff for navigation it also possesses a bottleneck as each employee can only communicate with one person at a time.

\textbf{Board of Directors}: This group can be split into several subgroups containing for instance, ''economy'' and ''scheduling and quality'', navigations on a hospital, can have a great impact on both  of these group, if and employee is distracted from his/her work, and they have multiply things to attend too, often a lost patient or a visitor will distract them, this will effect there work and may often end up in a stressed staff, while this effect there concentration and quality of there work, an unconsecrated employee can make mistakes, while there on a tight time schedule, and a unforeseen obstacle like a lost patient or a visitor, can push scheduling, and a result of this can cost the hospital a lot of time, money and stressed staff. 


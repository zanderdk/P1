%!TEX root = ../../Master.tex
\section{Problems tied to the systems} % (fold)
\label{sec:Problems_tied_to_the_systems}

\paragraph{Phone}
Visitors are able to call the hospital's main number, and ask questions to a live operator. This can be done from any phone but there is no guarantee that a live operator is available.

\paragraph{Signs}
There are signs placed that mark different areas of the hospital. "Main entrance" and "Ambulance entrance" signs can be used to mark key places that the visitor can navigate from\cite{art_Osborne}.
Interior signs are also places in the hospital. These painted to the walls, affixed to doors or windows and others hanging from the ceiling. These describe what room or hall the visitor is in. Some of these signs also contain information about where the different areas of the hospital can be located, often marked by a line of text followed up by a arrow pointing in a specified direction. If the hospital is made up by multiple buildings, they can be numbered in order to navigate people to specific buildings.

\paragraph{Maps}
Maps offers a top-down view of the hospital with all the different locations marked by text or colour\cite{art_Osborne}. The map can be painted on a wall or found in a compact version meant to be carried around. The stationary maps often have a red dot that marks where the reader is. By knowing the current position, the visitor should be able to navigate with more ease as they won't have to look for something recognizable. If the building has multiple floors, the map will be split up into sections in order to cover all the rooms.

\paragraph{The receptionist}
The receptionist can answer questions from the visitors. Very much like the phone, but with some key differences. The phone is more accessible as the one calling doesn't need to be at the hospital. The receptions has an advantage of being more precise. There will be less confusion as body language can be used in the answering of the question.

\paragraph{Colour coding}
Coloured stripes across the wall or floor that leads to the different areas of the hospital. In the entrance hall the visitors will see a wall with some lines of text with a coloured stripe behind it. One line might say "recovery" and if followed, will lead to the recovery department. Some departments also have an entire theme in a certain colour. In this way, it might become easier for some people to navigate the next time they visit, if they can remember the colours representing the department. 

\paragraph{Porter}*
The porter helps patients and visitors get around at the hospital. The porter helps visitors find available beds if they have to stay at the hospital overnight \cite{ugd_port}. 

% section Problems_tied_to_the_systems(end)
%!TEX root = ../../Master.tex

\section{Technology} \label{tech}

Before anyone can develop an optimal solution to a problem, it is nessesary to understand the technology at ones disposal. But since understanding technology is time consuming, it is important to analyse which to focus on. In the following section, we have selected the technologies that we estimated had the most relevance to our problem, and analyzed how they might be of use in a possible software solution. It is important to note, that we have only analyzed technologies, that supports a possible solution to the entirety of our problem, with a few exception. These exceptions have been included because their hold a prominent role in a common optimal solution to a similar problem.

The analyzis of the technologies followed a certain simple method. We describe the different existing digital technology, and analyze their relevance to our initating problem. We then outline the function of a given technology, and dertermine its relevance by examining why the technology is in use, and specifically which traits makes the technology relevant for our problem. We subsequently describe any undesirable traits which will decrease the usefulness in a possible solution.

\subsection{Method} \label{sub:techmethod}
This method gives a good overview of the usefulness of the given technology, without going into much detail. Details, that are deemed important for understanding a solution or is not otherwise easily optainable, will included in a later theoretical chapter.

\kanote{Der efterfølgende stykke har jeg ikke skrevet. Jeg synes godt vi kan beholde meningen, og så bare skrive det om, så vi indlejrer definitionerne, og ikke referer til andre steder? OGSÅ: ideen om at kort beskrive aspekterne af navigation er godt, men skal det stå her?}
In order to navigate individuals indoor, it is important to always find the most optimal route based on the users prerequisites. This makes wayfinding technologies very important to our project, and we will cover this in \cref{sub:way}. For any wayfinding technique two parameters is required: a start and an end location. The start location can be determined by positioning the individual, this is done using the methods described in \cref{sub:pos}. Positioning have some requirements that the infrastructure has to comply with, we will define those requirements in \cref{sub:infra}.


%Teknologier der kan medtages (HVAD & HVORFOR, IKKE HVORDAN): GPS, RFID, QR-kode, Google Maps, Smartphones, Tablets, PDA, Handheld computing/tracking device, accellerometer, WiFi og andre relevante ting.


%!TEX root = ../../Master.tex

\subsection{Wayfinding}
 
%!TEX root = ../../Master.tex


\subsection{Positioning}\label{sub:pos}

  Triangulation is a method used for estimating the position of an object. The method uses the properties of triangles. Triangulation has two derivatives: literation and angulation.

 \subsubsection{Lateration}

  Lateration is commonly used technique for positioning.  Lateration uses the distance between known locations and and the point to be determined, to estimate the location.\cite{tri_lateration}  There are two types of lateration two dimensional and three dimensional. The two dimensional are use in robots that navigates in only one plane, like robot vacuum cleaners. 
  lateration in three dimensions are used in GPS positioning, and other practical applications for positioning.

  \subsubsection{Angulation}

  Angulation uses the technique called Angle of Arrival (AOA), which can compute the location of a source. The parameters needed are as minimum two relative angles between a remote point and the source, and the distance between the remote points. The source is at the location where the lines formed by the angle direction lines intercept. 

  In 2D, as few as two remote points are needed, given the source is not directly in between the remote points. However to improve accuracy and remove ambiguity  another remote point is needed. In 3D, as few as three remote points are needed. However as with 2D, another remote point improves accuracy and removes ambiguity. \cite{Liu2007, Sun2009, Boontrai2009}


  \subsubsection{Location Fingerprinting}


  Location fingerprinting is another techinque used for positioning. It refers to the algorithms that first collect characteristics (fingerprints) of a scene and then matches these a priori characteristics with the characteristics of the location the source is in, to determine the source' location.

  There are two stages for location fingerprinting. The offline state where a site survey is performed in the environment. The characteristics of known locations are stored in databases. In the online state, these a priori characteristics are used to match the location of the source.

  The main challenge of location fingerprinting is the possibility of variating signal strengths corrupting characteristics data.

  \subsubsection{Summary}


  All of the above described technologies need remote points to find a position. As described in \cref{sub:infra}, many hospitals have WiFi hotspots located around the hospital. These WiFi hotspots can be used as the remote points, given that the coverage of the network is good.

%!TEX root = ../../Master.tex
\subsection{Existing infrastructure in a hospital} \label{sub:infra}

\subsubsection{WiFi}

Most hospitals have good WiFi coverage around their cadastral. A visit to "Sygehus Nord" in Aalborg revealed that multiple locations around the hospital had a sufficient amount of WiFi hotspots in range, in order to use several of the positioning techniques covered in \cref{sub:pos}.

% \begin{figure}
% \centering
% 	\subcaptionbox{A cat\label{cat}}
% 	{\includegraphics[width=\textwidth]{wifi_sygehus_nord1.png}}

% 		\subcaptionbox{A cat\label{cat}}
% 	{\includegraphics[width=\textwidth]{wifi_sygehus_nord1.png}}

% \end{figure}

\begin{figure}[htb]
	\begin{center} 
		\subcaptionbox{Graph of signal strength grouped by channels. Location A}
		{\includegraphics[width=0.33\textwidth]{wifi_sygehus_nord1.png}}
		\quad
		\subcaptionbox{Graph of signal strength grouped by channels. Location B}
		{\includegraphics[width=0.33\textwidth]{wifi_sygehus_nord2.png}}
		\quad
		\subcaptionbox{List of WiFi networks}
		{\includegraphics[width=0.33\textwidth]{wifi_sygehus_nord3.png}}
%\subfigure[label2][s]{\includegraphics[]{wifi_sygehus_nord1.png}}
%\subfigure[label3][s]{\includegraphics[]{wifi_sygehus_nord1.png}}
\end{center}
\caption{Analysis of wireless networks at Sygehus Nord Aalborg}
\label{fig:stefanResidual}
\end{figure}
%!TEX root = ../../Master.tex

\section{Technology} \label{tech}

% \kanote{Der efterfølgende stykke har jeg ikke skrevet. Jeg synes godt vi kan beholde meningen, og så bare skrive det om, så vi indlejrer definitionerne, og ikke referer til andre steder? OGSÅ: ideen om at kort beskrive aspekterne af navigation er godt, men skal det stå her?}
% In order to navigate individuals indoor, it is important to always find the most optimal route based on the users prerequisites. This makes wayfinding technologies very important to our project, and we will cover this in \cref{sub:way}. For any wayfinding technique two parameters is required: a start and an end location. The start location can be determined by positioning the individual. This is done using the methods described in \cref{sub:pos}. Positioning have some requirements that the infrastructure has to comply with, we will define those requirements in \cref{sub:infra}.

Before anyone can develop an optimal solution to a problem, it is necessary to understand the technology at one's disposal. But since understanding technology is time consuming, it is important to analyse which to focus on. In the following section, we have selected the technologies that we estimated have the most relevance to our problem, and analysed how they might be used in a possible software solution. It is important to note, that we have only analysed technologies, that support a possible solution to the entirety of our problem, with a few exception. These exceptions have been included because their hold a prominent role in a common optimal solution to a similar problems.

The analysis of the technologies follows a certain simple method. We describe the different existing digital technology, and analyse their relevance to our initiating problem. We then outline the function of a given technology, and determine its relevance by examining why the technology is in use, and specifically which traits makes the technology relevant for our problem. We subsequently describe any undesirable traits which will decrease the usefulness in a possible solution.

\subsection{Method} \label{sub:techmethod}
This method gives a good overview of the usefulness of the given technology, without going into much detail. Details, that are deemed important for understanding a solution or is not otherwise easily obtainable, will be included in a later theoretical chapter.




%Teknologier der kan medtages (HVAD & HVORFOR, IKKE HVORDAN): GPS, RFID, QR-kode, Google Maps, Smartphones, Tablets, PDA, Handheld computing/tracking device, accellerometer, WiFi og andre relevante ting.

%%!TEX root = ../../Master.tex
\sinote{Afgræns i afgrænsningen; vi vil kun arbejde med software med øje til handheld devices}

\subsection{Handheld Devices} % (fold)
\label{sub:device}

% subsection subsection_name (end)
 	 
Handheld computing/tracking devices, such as smartphones, tablets and PDAs, are all relatively small and users are able to carry them around. Such devices have a screen, with the exception of some older PDAs, and a Wi-Fi or Bluetooth connection module, that allows them to send and/or receive data. This data can then be projected onto the screen, to inform the user.

%The PDA functions as a personal digital assistant, who will keep track of the user's calender or have a calculator program. Two popular PDA's are the iPod touch and BlackBerrie which are still in use. 
%The Smartphone are much alike the PDA, but have other features such as the ability to receive or make phone calls. Smartphones also a varity of apps that allows 3rd party programs to be installed. Such programs could be digital games or social media programs like Facebook or Twitter. Two popular smartphone series would be the iPhoneor the Samsung galaxy.
%The Tablet is bigger in size compared to the PDA and smartphone, and won't fir in the users pocket. They serve almost the same purpose as the smartphones but have more computing power and memory storage. The tablets are used to satisf the needs that laptop covers that a smartphone will not do competent enough. Such needs could be to surf the world wide net, which many do on their laptop or home PC. This can be done on a smartphone but the screen is often seen as too small. The tablet have a bigger screen and thereofre provides a better experience. Two popular tablet would be the iPad and the Surface. 
Devices like smartphones and tablets are already in use, and they are very popular. This means that for many users, a solution with a downloadable program or application, would be easily adapted. Other navigation programs/applications that they have already used, could have similarities with the solution used in this project, which will ease apdaption even more. These devices are also very portable and can be taken to the hospital without much effort. They can also be used to render a map or otherwise assist the user with navigation, for instance through text, sound or pictures.

For these devices to be used as intended, they requires some prior basic digital knowledge. In order to use the device to navigate trough a given method, the user would have to start the device and navigate to the appropriate application in order to get started. This could potentially hinder some users as this could be confusing to them.

One of the caveats of handheld devices is the limited battery capacity. The battery time of a device depends on many factors including screen-on time, processor usage, radio modems etc. As an example a widely used handheld device has a rated \enquote{internet use} battery time of up to 10 hours. \cite{Apple}

When designing the navigation system the limited battery capacity of these handheld devices must be taken into account.
%The modern devices are also fairly expensive which could make them less attractive to users.

%!TEX root = ../../Master.tex
\sinote{Afgræns i afgrænsningen; vi vil kun arbejde med software med øje til handheld devices}

\subsection{Handheld Devices} % (fold)
\label{sub:device}

% subsection subsection_name (end)
 	 
Handheld computing/tracking devices, such as smartphones, tablets and PDAs, are all relatively small and users are able to carry them around. Such devices have a screen, with the exception of some older PDAs, and a Wi-Fi or Bluetooth connection module, that allows them to send and/or receive data. This data can then be projected onto the screen, to inform the user.

%The PDA functions as a personal digital assistant, who will keep track of the user's calender or have a calculator program. Two popular PDA's are the iPod touch and BlackBerrie which are still in use. 
%The Smartphone are much alike the PDA, but have other features such as the ability to receive or make phone calls. Smartphones also a varity of apps that allows 3rd party programs to be installed. Such programs could be digital games or social media programs like Facebook or Twitter. Two popular smartphone series would be the iPhoneor the Samsung galaxy.
%The Tablet is bigger in size compared to the PDA and smartphone, and won't fir in the users pocket. They serve almost the same purpose as the smartphones but have more computing power and memory storage. The tablets are used to satisf the needs that laptop covers that a smartphone will not do competent enough. Such needs could be to surf the world wide net, which many do on their laptop or home PC. This can be done on a smartphone but the screen is often seen as too small. The tablet have a bigger screen and thereofre provides a better experience. Two popular tablet would be the iPad and the Surface. 
Devices like smartphones and tablets are already in use, and they are very popular. This means that for many users, a solution with a downloadable program or application, would be easily adapted. Other navigation programs/applications that they have already used, could have similarities with the solution used in this project, which will ease apdaption even more. These devices are also very portable and can be taken to the hospital without much effort. They can also be used to render a map or otherwise assist the user with navigation, for instance through text, sound or pictures.

For these devices to be used as intended, they requires some prior basic digital knowledge. In order to use the device to navigate trough a given method, the user would have to start the device and navigate to the appropriate application in order to get started. This could potentially hinder some users as this could be confusing to them.

One of the caveats of handheld devices is the limited battery capacity. The battery time of a device depends on many factors including screen-on time, processor usage, radio modems etc. As an example a widely used handheld device has a rated \enquote{internet use} battery time of up to 10 hours. \cite{Apple}

When designing the navigation system the limited battery capacity of these handheld devices must be taken into account.
%The modern devices are also fairly expensive which could make them less attractive to users.

%!TEX root = ../../Master.tex
\subsection{Google Maps}
Google Maps is a web mapping service hosted by Google inc. The application is used by some of Google's other products such as Google Transit\cite{Goo_transist}. 

When a user is using on the Google maps application, all information will be downloaded to their device. Information is downloaded when they search on a location or when they drag the map around\cite{Goo_input}. The satellite images are acquired from other companies like Tele Atlas\cite{Goo_Tele} or Zenrin\cite{Goo_Zenrin} and then overlayed with Ground truth\cite{Goo_GT}.
The map will be shown with a top down view like many other maps and will offer satellite images, road maps or dynamic maps. It is also possible to go into Street View which is another Google tool, from where the viewpoint is from the street level\cite{Goo_street}.

Google maps can be used to plan routes, show earthquakes and even lets you swim with turtles\cite{Goo_Turtle}.

Google maps is relevant for this project as they have made it able to show maps inside buildings, as seen at Aalborg Universitet, Cassiopeia\cite{Goo_Indoor}. This system could be implanted at the hospital so it would show the different floors and have them all mapped. Instead of using a GPS, Google Maps Indoor uses Wi-Fi hotspots and telephone towers to locate their target\cite{Goo_Indoor}.

It is not certain that every complex has enough Wi-Fi hotspots or telephone towers nearby. Therefore this may not be a general solution.

\sinote{Beskriv under WiFi at det er strømkrævende at holde en WiFi sender kørende/sendende}

\sinote{Mangler konklusion af hvilke lokationsbaserede services vi har fundet frem til virker bedst i vores situation}

\sinote{Beskriv hvad location fingerprinting er. Skriv en indledning til hvorfor vi skriver om RFID, WIFI osv.}

 
%%!TEX root = ../../Master.tex


\subsection{Positioning}\label{sub:pos}

  Triangulation is a method used for estimating the position of an object. The method uses the properties of triangles. Triangulation has two derivatives: literation and angulation.

 \subsubsection{Lateration}

  Lateration is commonly used technique for positioning.  Lateration uses the distance between known locations and and the point to be determined, to estimate the location.\cite{tri_lateration}  There are two types of lateration two dimensional and three dimensional. The two dimensional are use in robots that navigates in only one plane, like robot vacuum cleaners. 
  lateration in three dimensions are used in GPS positioning, and other practical applications for positioning.

  \subsubsection{Angulation}

  Angulation uses the technique called Angle of Arrival (AOA), which can compute the location of a source. The parameters needed are as minimum two relative angles between a remote point and the source, and the distance between the remote points. The source is at the location where the lines formed by the angle direction lines intercept. 

  In 2D, as few as two remote points are needed, given the source is not directly in between the remote points. However to improve accuracy and remove ambiguity  another remote point is needed. In 3D, as few as three remote points are needed. However as with 2D, another remote point improves accuracy and removes ambiguity. \cite{Liu2007, Sun2009, Boontrai2009}


  \subsubsection{Location Fingerprinting}


  Location fingerprinting is another techinque used for positioning. It refers to the algorithms that first collect characteristics (fingerprints) of a scene and then matches these a priori characteristics with the characteristics of the location the source is in, to determine the source' location.

  There are two stages for location fingerprinting. The offline state where a site survey is performed in the environment. The characteristics of known locations are stored in databases. In the online state, these a priori characteristics are used to match the location of the source.

  The main challenge of location fingerprinting is the possibility of variating signal strengths corrupting characteristics data.

  \subsubsection{Summary}


  All of the above described technologies need remote points to find a position. As described in \cref{sub:infra}, many hospitals have WiFi hotspots located around the hospital. These WiFi hotspots can be used as the remote points, given that the coverage of the network is good.

%%!TEX root = ../../Master.tex
\subsection{Existing infrastructure in a hospital} \label{sub:infra}

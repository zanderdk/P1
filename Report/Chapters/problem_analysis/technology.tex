%!TEX root = ../../Master.tex

\section{Technology} \label{tech}

% \kanote{Der efterfølgende stykke har jeg ikke skrevet. Jeg synes godt vi kan beholde meningen, og så bare skrive det om, så vi indlejrer definitionerne, og ikke referer til andre steder? OGSÅ: ideen om at kort beskrive aspekterne af navigation er godt, men skal det stå her?}
% In order to navigate individuals indoor, it is important to always find the most optimal route based on the users prerequisites. This makes wayfinding technologies very important to our project, and we will cover this in \cref{sub:way}. For any wayfinding technique two parameters is required: a start and an end location. The start location can be determined by positioning the individual. This is done using the methods described in \cref{sub:pos}. Positioning have some requirements that the infrastructure has to comply with, we will define those requirements in \cref{sub:infra}.

Before anyone can develop an optimal solution to a problem, it is necessary to understand the technology at one's disposal. But since understanding technology is time consuming, it is important to analyse which to focus on. In the following section, we have selected the technologies that we estimated have the most relevance to our problem, and analysed how they might be used in a possible software solution. It is important to note, that we have only analysed technologies, that support a possible solution to the entirety of our problem, with a few exception. These exceptions have been included because their hold a prominent role in a common optimal solution to a similar problems.

The analysis of the technologies follows a certain simple method. We describe the different existing digital technology, and analyse their relevance to our initiating problem. We then outline the function of a given technology, and determine its relevance by examining why the technology is in use, and specifically which traits makes the technology relevant for our problem. We subsequently describe any undesirable traits which will decrease the usefulness in a possible solution.

\subsection{Method} \label{sub:techmethod}
This method gives a good overview of the usefulness of the given technology, without going into much detail. Details, that are deemed important for understanding a solution or is not otherwise easily obtainable, will be included in a later theoretical chapter.




%Teknologier der kan medtages (HVAD & HVORFOR, IKKE HVORDAN): GPS, RFID, QR-kode, Google Maps, Smartphones, Tablets, PDA, Handheld computing/tracking device, accellerometer, WiFi og andre relevante ting.

%!TEX root = ../../Master.tex
\sinote{Afgræns i afgrænsningen; vi vil kun arbejde med software med øje til handheld devices}

\subsection{Handheld Devices} % (fold)
\label{sub:device}

% subsection subsection_name (end)
 	 
Handheld computing/tracking devices, such as smartphones, tablets and PDAs, are all relatively small and users are able to carry them around. Such devices have a screen, with the exception of some older PDAs, and a Wi-Fi or Bluetooth connection module, that allows them to send and/or receive data. This data can then be projected onto the screen, to inform the user.

%The PDA functions as a personal digital assistant, who will keep track of the user's calender or have a calculator program. Two popular PDA's are the iPod touch and BlackBerrie which are still in use. 
%The Smartphone are much alike the PDA, but have other features such as the ability to receive or make phone calls. Smartphones also a varity of apps that allows 3rd party programs to be installed. Such programs could be digital games or social media programs like Facebook or Twitter. Two popular smartphone series would be the iPhoneor the Samsung galaxy.
%The Tablet is bigger in size compared to the PDA and smartphone, and won't fir in the users pocket. They serve almost the same purpose as the smartphones but have more computing power and memory storage. The tablets are used to satisf the needs that laptop covers that a smartphone will not do competent enough. Such needs could be to surf the world wide net, which many do on their laptop or home PC. This can be done on a smartphone but the screen is often seen as too small. The tablet have a bigger screen and thereofre provides a better experience. Two popular tablet would be the iPad and the Surface. 
Devices like smartphones and tablets are already in use, and they are very popular. This means that for many users, a solution with a downloadable program or application, would be easily adapted. Other navigation programs/applications that they have already used, could have similarities with the solution used in this project, which will ease apdaption even more. These devices are also very portable and can be taken to the hospital without much effort. They can also be used to render a map or otherwise assist the user with navigation, for instance through text, sound or pictures.

For these devices to be used as intended, they requires some prior basic digital knowledge. In order to use the device to navigate trough a given method, the user would have to start the device and navigate to the appropriate application in order to get started. This could potentially hinder some users as this could be confusing to them.

One of the caveats of handheld devices is the limited battery capacity. The battery time of a device depends on many factors including screen-on time, processor usage, radio modems etc. As an example a widely used handheld device has a rated \enquote{internet use} battery time of up to 10 hours. \cite{Apple}

When designing the navigation system the limited battery capacity of these handheld devices must be taken into account.
%The modern devices are also fairly expensive which could make them less attractive to users.

%!TEX root = ../../Master.tex
\subsection{Google Maps}
Google Maps is a web mapping service hosted by Google inc. The application is used by some of Google's other products such as Google Transit\cite{Goo_transist}. 

When a user is using on the Google maps application, all information will be downloaded to their device. Information is downloaded when they search on a location or when they drag the map around\cite{Goo_input}. The satellite images are acquired from other companies like Tele Atlas\cite{Goo_Tele} or Zenrin\cite{Goo_Zenrin} and then overlayed with Ground truth\cite{Goo_GT}.
The map will be shown with a top down view like many other maps and will offer satellite images, road maps or dynamic maps. It is also possible to go into Street View which is another Google tool, from where the viewpoint is from the street level\cite{Goo_street}.

Google maps can be used to plan routes, show earthquakes and even lets you swim with turtles\cite{Goo_Turtle}.

Google maps is relevant for this project as they have made it able to show maps inside buildings, as seen at Aalborg Universitet, Cassiopeia\cite{Goo_Indoor}. This system could be implanted at the hospital so it would show the different floors and have them all mapped. Instead of using a GPS, Google Maps Indoor uses Wi-Fi hotspots and telephone towers to locate their target\cite{Goo_Indoor}.

It is not certain that every complex has enough Wi-Fi hotspots or telephone towers nearby. Therefore this may not be a general solution.

%!TEX root = ../../Master.tex
\subsection{Location Fingerprinting}
In location fingerprinting, there are to main tecniques that have been choosen for this project. It is "QR-code" and "RFID".
Location fingerprinting is method of recognizing devices by giving them a electronic fingerprint, such as a unique id.

\paragraph{QR-code} % (fold)
QR-codes are barcodes that returns data when scanned. The code is represented by black boxes on a white square grid background. All by how the boxes are arranged, the code will represent different data.

In order to read the data, a scanner has to be used. A camera from a smartphone can be used for this purpose, as long there are an app that supports this particular format. Such apps can be found on nearly all smartphones\cite{QR_smart}. The QR-code can be used to lead users to websites, and are used often in different kind of commercial ways\cite{QR_url}.

It is easy to read the QR-code if you already have an app installed on your smartphone. It is also easy to make your own codes as Google have released a free tool to generate them\cite{QR_Google}.It is also rather easy to set up a QR-code system in a effective way\cite{QR_easy}.

The relevance for this project is how it will take little to no time to set up\cite{QR_rel1}, and how some people already knows how use them as they are widely spread\cite{QR_spread}. Even if a code is damaged it will still be usable, all up to 30\% of the code can be missing\cite{QR_dama}. This also means it will be possible to implant images in the code and still have it working\cite{QR_image}.

A reason for not to use the codes could be that some users might exploit the system. If they print their own codes with links to certain websites it could be a very unpleasant experience for the other users\cite{QR_urlbad}. In worst case scenarios they could be used to steal personal information\cite{QR_information}.
In order for the codes to be efficient, they also have to placed at many locations at the hospital. The users should always be able to find them when they want to use them. This could leave to frustration if they seems to be nowhere to find for the user.


\paragraph{RFID}
RFID (Radio Frequency Identification) is a radio technology that makes it possible, just like barcodes, to identify different objects. the main difference between these two technologies is that the barcode is a line of sight technology, and RFID is wireless, as long as it is within visible distance. The readable distance varies depending on the type of RFID tag. The information exchanged can be in many different formats such as text, numbers, audio or video. It depends on the size of the RFID tag built-in memory. 

There are, different RFID tags, as they can be passive, active or battery-assisted. Tags can be read-only "one time read" or read-write "multiply times", have a serial number, or can be plank for the owner to write something on it. To put it into perspective to our project, a read-only tag can be used to find an object such as a printer that does not change its name or property. Read-write tag, can be used to track patients, the information can chance the name and CPR number to match the person who wears it.  


%!TEX root = ../../Master.tex
\subsection{Accelerometer}

The Accelerometer is a transducer that measures acceleration. The acceleration could be static if the device is at hold an only affected by gravity, or dynamic if it moves around\cite{acc_engi}. Accelerometers is used in many different ways and are also used in a lot of different applications, such the PlayStation 3 DualShock 3 remote\cite{acc_ps3} or in a Nokia 5500 sport\cite{acc_nokia}.


The accelerometer is able to detect vibrations, change of altitude or direction\cite{acc_engi}. Some accelerometers works by relying on the piezoelectric effect that delivers electronic output from a variety of crystals inside the device\cite{acc_piezo}. This electronic output can then be read by a computer and used to generate information about the state of the accelerometer.


The accelerometer is among other used because it can reliable be applied in car GPS systems. It uses the technique dead reckoning that estimates ones position based on previously coordinates and current vector movement\cite{acc_dead}. The accelerometer has also proven itself useful in some science aspects, such as earthquake investigations or monitoring active volcanoes\cite{acc_vulkan}. 


The accelerometer is relevant for this project, as it can be used to position the users if it is incorporated in a device that is going to be carried along side the current user. In this way it could be used to see if the user is following the generated path or if he/she have moved outside of it. The accelerometers that could be used for this project could be the LIS302DL as it is rather cheap and compact\cite{acc_price,acc_lis302dl}. This accelerometer have already been incorporated in the original iPhone\cite{acc_iPhone}.


A downside with the accelerometer is that it can not stand on its own. The accelerometer is unable to position itself and will therefore need to work alongside another system, such as a GPS or wifi and etcetera. 


%!TEX root = ../../Master.tex

\subsection{GPS}

Global Positioning System, GPS is a satellite-based positioning system. Satellites emits a signal, a time-stamp and current position. A GPS receiver measures the distance from itself to any satellite, by the time it took to receive the signal at the speed of light. When the positions and distances of multiple of satellites is known, its possible determine the GPS receivers position by trilateration \cite{Dempster2013}. It is possible to use GPS all over the world as long as there a clear line of sight to the satellites. Due to the signals not being able to pass through walls and roofs, means that it is suitable for indoor use \cite{GPS_about}.

\subsection{Summary}
This section analysed which technology were thought to be relevant for the project solution. Wi-Fi and accelerometer were chosen as the best tools for positioning. As seen in \cref{orgwifi} Wi-Fi is already implemented in most hospitals, reducing setup costs. Wi-Fi can be used for positioning inside a building, and an accelerrator can be used to reduce Wi-Fi power consumption because it can track in which direction the user is heading.
 
%%!TEX root = ../../Master.tex


\subsection{Positioning}\label{sub:pos}

  Triangulation is a method used for estimating the position of an object. The method uses the properties of triangles. Triangulation has two derivatives: literation and angulation.

 \subsubsection{Lateration}

  Lateration is commonly used technique for positioning.  Lateration uses the distance between known locations and and the point to be determined, to estimate the location.\cite{tri_lateration}  There are two types of lateration two dimensional and three dimensional. The two dimensional are use in robots that navigates in only one plane, like robot vacuum cleaners. 
  lateration in three dimensions are used in GPS positioning, and other practical applications for positioning.

  \subsubsection{Angulation}

  Angulation uses the technique called Angle of Arrival (AOA), which can compute the location of a source. The parameters needed are as minimum two relative angles between a remote point and the source, and the distance between the remote points. The source is at the location where the lines formed by the angle direction lines intercept. 

  In 2D, as few as two remote points are needed, given the source is not directly in between the remote points. However to improve accuracy and remove ambiguity  another remote point is needed. In 3D, as few as three remote points are needed. However as with 2D, another remote point improves accuracy and removes ambiguity. \cite{Liu2007, Sun2009, Boontrai2009}


  \subsubsection{Location Fingerprinting}


  Location fingerprinting is another techinque used for positioning. It refers to the algorithms that first collect characteristics (fingerprints) of a scene and then matches these a priori characteristics with the characteristics of the location the source is in, to determine the source' location.

  There are two stages for location fingerprinting. The offline state where a site survey is performed in the environment. The characteristics of known locations are stored in databases. In the online state, these a priori characteristics are used to match the location of the source.

  The main challenge of location fingerprinting is the possibility of variating signal strengths corrupting characteristics data.

  \subsubsection{Summary}


  All of the above described technologies need remote points to find a position. As described in \cref{sub:infra}, many hospitals have WiFi hotspots located around the hospital. These WiFi hotspots can be used as the remote points, given that the coverage of the network is good.
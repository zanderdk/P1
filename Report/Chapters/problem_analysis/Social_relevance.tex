%!TEX root = ../../Master.tex
\section{Social Relevance}

Technology plays a major role in today's modern society and everywhere you go, you will stumble upon some form of technology. Today's society would not exist without the technological advancements which man has made in the past. Despite this fact, not all technological advancements remain relevant and not all technologies have a use in a modern society. An example of this would be the steam pump. It was once a very important invention and to a degree it still is because it laid the foundation for more advanced technology. However it became obsolete as more advanced technology has made a more advanced pump. 

Hence the relevancy to society of a certain technology is determined by the use it has, and whether it can compete with existing technologies in its own field.

\subsection{Navigating Today's World}
It can be a daunting task to find your way around today's modern society. Past one's comfort zone the roads may seem infinite and navigating them, an impossible task. This was previously, and still is, alleviated with the use of maps, however maps will not dynamically update your current position and reading a map can be very difficult for the inexperienced. However, in the 1980's, the world saw the advent of GPS\cite{gps_advent}. GPS navigation would improve upon navigation via maps as it plans the route for the user, while at the same time dynamically updates the user position, which removed the need for the user to keep track of their position. This made the need for reading a map and planning a route superfluous.

These improvements over the traditional map meant that GPS navigation would be capable of competing with the existing technology at the time (i.e. maps, compass etc.). GPS navigation has remained a relevant technology to this day because navigating around the world is just as necessary and challenging now as it has ever been.

\subsection{Navigating indoors}
GPS navigation helped make outdoor navigation easier, but the system does not work indoors, due to limitations with the signal and accuracy\cite{gps_tech}. With facilities getting larger and more complex every day, the need for indoor navigation has never been greater.

While outdoor navigation saw the advent of GPS navigation, indoor navigation has had to rely on old fashioned navigation where the user has to manually position and navigate themselves, as described in \cref{sec:anal_nav}. With the increasing use of GPS navigation, which requires little to no user action, it would seem anything but likely that people have become more skilled at manual navigation. This poses the problem of how people can navigate reliably indoors.

What makes this a problem is the fact, that if individuals cannot get to where they need to be, they cannot perform the task they set out to do. This leads to frustration for those involved, be it the visitor at a hospital or the patient awaiting a visit. While navigation may seem simple to many it is imperative to acknowledge that it becomes easier with knowledge of the area and newcomers most definitely will struggle to find their way.

Many solutions to this have been proposed and developed over time, however it remains evident that none of the solutions have managed to solve the problem to satisfaction, as proved by the continued attempts to solve the problem\cite{skejby_attempt}.

\subsection{Relevance of Better Indoor Navigation}
In recent years the Danish government has restructured and made several cutbacks to the national healthcare system in attempts to save money\cite{cutback_danNHS}, however it was done on the premise that it would not compromise on the quality of the healthcare provided. This means that the Danish hospitals now have to provide the same level of care for a fraction of the cost and so efficiency improvement has been a key point in the aforementioned restructurings. One of the ways the Danish hospitals have had to deal with the cutbacks is by laying off employees\cite{cutback_firing}, this means the remaining staff (nurses etc.) have to do more in the same amount of time, so effectivization of the staff's time is crucial.

A 2004 study\cite{twaste_2004}, researching the role of the physical environment in modern hospitals, found that in a 300-bed hospital, the staff spent around 4500 hours per year helping patients and visitors find their way around the hospital. This amounts to around 20 hours a day of time wasted, time that could potentially be saved by better indoor navigation.

With so much focus on effectivity and saving money, this is an issue that must be considered relevant, and should a proposed solution be able to track positions dynamically it would be able to save even more time as the study also shows that a large amount of time is wasted by the staff looking for medications, patients etc.

\subsection{Summary}
The solution to this problem may not be simple as many complexities arise when dealing with different prerequisites and goals, however it is not a problem that should be overlooked as the potential gains are substantial and amount to both money and time saved and it should save many frustrations for the people as a whole.
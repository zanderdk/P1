\section{Social relevance}
Technology plays a huge role i todays modern society and everywhere you go you will stumble upon some form of technology; todays society cannot exist without the technological advancements which man has made in the past.
Despite this fact, not all technological advancements remain relevant and not all technologies have a use in a modern society. An example of this would be the steam pump, it was once a very important invention and to a degree it still is because it laid the foundation for more advanced technology, however it has become obsolete as the more advanced technology has made a more advanced pump. Hence the relevance of a technology is as much determined by the use it has in society and whether it can compete with existing technologies in its own field. \\
It can be a daunting task to find your way around in todays modern society, not so much in your local area but past ones comfort zone the roads may seem infinite and navigating them an impossible task. This was previously, and still is, alleviated with the use of maps, however maps will not dynamically update your current position and reading a map can be very difficult for the inexperienced. However, in the 1980's [1], the world saw the advent of GPS. GPS navigation would be able to improve upon navigating via maps as it dynamically updated the user position which removed the need for the user to keep track of their position, and it removed the need for reading a map and planning a route as the navigation system could tell you where to go and when. These improvements over the simple map meant that GPS navigation would be very much capable of competing with the existing technology at the time (i.e. maps, compass etc.). GPS navigation has remained a relevant technology to this day because navigating around the world is just as nescessary now as it has ever been. \\
GPS navigation helped make outdoor navigation easier and simpler but the system does not work indoors, due to limitations with the signal and accuracy, but with facilities getting larger and more complex every day, the need for indoor navigation has never been greater. \\
While outdoor navigation saw the advent of GPS navigation, indoor navigation have had to rely on the old fashioned navigation where the user has to manually position and navigate themselves, and with the increasing usage of GPS navigation, which requires little to no user action, it would seem anything but likely that people have become more skilled at manual navigation. This poses the problem of how people can navigate reliably indoors? \\
What makes this a problem is the fact that if people cannot get to where they have to be, they cannot perform the tast they set out to do; this leads to frustration for those involved, be it the visitor at a hospital or the patient awaiting a visit, or the technician sent to perform maintenance or the people relying on that maintenance being carried out. While navigation may seem simple to many it is imperative to acknowledge that it becomes easier with knowledge of the area and newcomers most definitely will struggle to find their way \\
Many solutions to this have been proposed and developed over time, however it remains evident that none of the solutions have managed to solve the problem to satisfaction as it remains to this day.
\jonote{Sektionen er ikke færdig. Den indeholdt tidligere yderligere afsnit, men jeg valgte at fjerne disse da de ikke passede ind i konteksten. Der skal ligeledes indsættes fundne kilder som henvisninger i teksten}
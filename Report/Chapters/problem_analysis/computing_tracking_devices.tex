 % Handheld computing/tracking device:
 % 	 Smartphones
 % 	 Tablets
 % 	 PDA
 	 
 	 
 % 	 1 Hvad er det?
 % 	 ----Hvad gør det? (hvilken form for input/output?)
 % 	 2 Hvorfor bruges den?
 % 	 ----Hvorfor er den relevant?
 % 	 ----Hvorfor mangler den relevans?

\subsection{Handheld devices} % (fold)
\label{sub:device}

% subsection subsection_name (end)
 	 
Handheld computing/tracking devices such as smartphones, tablets and PDA's are all relative small handhelds that the user are able to carry around. Such devices have a screen (some old PDAs do not) and a Wi-Fi or Bluetooth connection module that allows them to send/receive data. This data can then be projected onto the screen, to inform the user.

%The PDA functions as a personal digital assistant, who will keep track of the user's calender or have a calculator program. Two popular PDA's are the iPod touch and BlackBerrie which are still in use. 
%The Smartphone are much alike the PDA, but have other features such as the ability to receive or make phone calls. Smartphones also a varity of apps that allows 3rd party programs to be installed. Such programs could be digital games or social media programs like Facebook or Twitter. Two popular smartphone series would be the iPhoneor the Samsung galaxy.
%The Tablet is bigger in size compared to the PDA and smartphone, and won't fir in the users pocket. They serve almost the same purpose as the smartphones but have more computing power and memory storage. The tablets are used to satisf the needs that laptop covers that a smartphone will not do competent enough. Such needs could be to surf the world wide net, which many do on their laptop or home PC. This can be done on a smartphone but the screen is often seen as too small. The tablet have a bigger screen and thereofre provides a better experience. Two popular tablet would be the iPad and the Surface. 
Devices like the smartphone and tablet are already in use and are very popular. This means that for some users, having a program or app on their own device, would be easy to adapt to. Other navigation apps/programs that they have already used, would maybe have some similarities with the solution used in this project, and will therefore be easy to adapt to. These devices are also very portable and can be taken to the hospital without much effort.  

These handheld computing/tracking devices can all be used to project a map or in other ways navigate the user, through text, symbols or pictures shown on the screen. They can be brought in most electronic stores and can be carried around in the users pocket (with the exception of the tablet which because of its bigger size would need a bag). 

These devices needs some basic digital know-how, in order to be used as intended. In order to use the device to navigate trough a given method, the user would have to start the device and navigate to the appropriate application in order to get started. This could potentially hinder some users as this could seem confusing for them. The modern devices are also fairly expensive which could make the less attractive to users.
%!TEX root = ../../Master.tex
\sinote{Afgræns i afgrænsningen; vi vil kun arbejde med software med øje til handheld devices}

\subsection{Handheld Devices} % (fold)
\label{sub:device}

% subsection subsection_name (end)
 	 
Handheld computing/tracking devices, such as smartphones, tablets and PDAs, are all relatively small and users are able to carry them around. Such devices have a screen, with the exception of some older PDAs, and a Wi-Fi or Bluetooth connection module, that allows them to send and/or receive data. This data can then be projected onto the screen, to inform the user.

%The PDA functions as a personal digital assistant, who will keep track of the user's calender or have a calculator program. Two popular PDA's are the iPod touch and BlackBerrie which are still in use. 
%The Smartphone are much alike the PDA, but have other features such as the ability to receive or make phone calls. Smartphones also a varity of apps that allows 3rd party programs to be installed. Such programs could be digital games or social media programs like Facebook or Twitter. Two popular smartphone series would be the iPhoneor the Samsung galaxy.
%The Tablet is bigger in size compared to the PDA and smartphone, and won't fir in the users pocket. They serve almost the same purpose as the smartphones but have more computing power and memory storage. The tablets are used to satisf the needs that laptop covers that a smartphone will not do competent enough. Such needs could be to surf the world wide net, which many do on their laptop or home PC. This can be done on a smartphone but the screen is often seen as too small. The tablet have a bigger screen and thereofre provides a better experience. Two popular tablet would be the iPad and the Surface. 
Devices like smartphones and tablets are already in use, and they are very popular. This means that for many users, a solution with a downloadable program or application, would be easily adapted. Other navigation programs/applications that they have already used, could have similarities with the solution used in this project, which will ease apdaption even more. These devices are also very portable and can be taken to the hospital without much effort. They can also be used to render a map or otherwise assist the user with navigation, for instance through text, sound or pictures.

For these devices to be used as intended, they requires some prior basic digital knowledge. In order to use the device to navigate trough a given method, the user would have to start the device and navigate to the appropriate application in order to get started. This could potentially hinder some users as this could be confusing to them.

One of the caveats of handheld devices is the limited battery capacity. The battery time of a device depends on many factors including screen-on time, processor usage, radio modems etc. As an example a widely used handheld device has a rated \enquote{internet use} battery time of up to 10 hours. \cite{Apple}

When designing the navigation system the limited battery capacity of these handheld devices must be taken into account.
%The modern devices are also fairly expensive which could make them less attractive to users.
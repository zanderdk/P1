%!TEX root = ../../Master.tex
\subsection{Accelerometer}\label{subs_acel}

The Accelerometer is a transducer that measures acceleration. The acceleration the accelerometer measures could be static if the device is at hold and only affected by gravity, or dynamic if it is moved around\cite{acc_engi}. Accelerometers is used in many different ways and are also used in a lot of different applications, such as the PlayStation 3 DualShock 3 remote\cite{acc_ps3} or in a older device like the Nokia 5500 sport\cite{acc_nokia}.


The accelerometer is able to detect vibrations, change of altitude or direction\cite{acc_engi}. Some accelerometers works by relying on the piezoelectric effect that delivers electronic output from a variety of crystals inside the device\cite{acc_piezo}. This electronic output can then be read by a computer, and used to generate information about the state of the accelerometer.


The accelerometer is among other used because it can reliable be applied in car GPS systems. It uses the technique dead reckoning that estimates ones position, based on previously coordinates and current vector movement\cite{acc_dead}. The accelerometer has also proven itself useful in some science aspects, such as earthquake investigations or monitoring active volcanoes\cite{acc_vulkan}. 


The accelerometer is relevant for this project, as it can be used to position the users if it is incorporated in a device that is carried along side the current user. In this way it could be used to see if the user is following the generated path or if he/she have moved outside of it. The accelerometers that could be used for this project could be the LIS302DL as it is rather cheap and compact\cite{acc_price,acc_lis302dl}. This accelerometer have already been incorporated in the original iPhone\cite{acc_iPhone}.


A downside with the accelerometer is that it can not stand on its own. The accelerometer is unable to position itself and will therefore need to work alongside another system, such as a GPS or wifi and etcetera. 
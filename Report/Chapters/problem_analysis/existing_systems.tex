%!TEX root = ../../Master.tex
\section{Existing systems} % (fold)
\label{sec:existing_systems}

% section existing_systems (end)

\textbf{Phone}
Visitors are able to call the hospitals main number, and ask questions to a live operator. This can be done from any phone but there is no insurance that a live operator is at disposal, in order to answer the phone.

\textbf{Signs}
There are placed signs that marks different areas of the hospital. "Main entrance" and "Ambulance entrance" these signs can be used to mark key places that the visitor can navigate from.
There are also placed interior signs in the hospital. Signs painted to the walls, signs affixed to doors or windows and others hang from the ceiling. These signs describes what rooms the reader are in or what the hall he is in. Some of these signs also contains information about where the different areas of the hospital can be located, often marked by a line of text followed up by a arrow pointing in a specified direction. If the hospital is made up by multiple buildings, they can be numbered in order to navigate people to specific buildings.

\textbf{Maps}
Maps offers a top-down view of the hospital with all the different locations marked by text or a colour. The map can be painted on a wall or found in a compact version meant to be carried around. The stationary maps often have a red dot that points out were the reader is. By knowing the current positions, the visitor should be able to navigate with more ease as they wont have to look for something they recognize. If the building have multiple floors, the map will be split up into sections, in order to cover all the rooms.

\textbf{The receptionist}
The receptionist will answer questions from the visitors. Very much like the phone, but with some key differences. The phone is more accessible as the one who calls does not need to be at a certain location. The receptions has an advantage for being more precise. There will be less confusion as body language can be used in the answering of the question.

\textbf{Colour coding}
Coloured stripes across the wall or floor that leads to the different areas of the hospital. In the entrance hall the visitors will se a wall with some lines of text with a coloured stripe behind it. One line might say "recovery" and if followed, will lead to the recovery department. Some departments also have an entire theme in a certain colour. In this way, it might become easier for some people to navigate the next time they visit, if they can remember the colours for a department. 

\textbf{Porter}*
The porter helps patients and visitors get around at the hospital. The porter helps visitors find available beds if they have to stay at the hospital overnight. 

Sources:
http://www.healthliteracy.com/article.asp?PageID=3756
http://da.wikipedia.org/wiki/Port%C3%B8r
Laila kirkeby Matthiassen - Intensiv sygeplejske

*A porter is a service employees who is employed at hospitals, hospital pharmacies and at nursing homes. The porter deals with a wide fan of practical and technical tasks, such as transport of patients, medicine, mail and journals between the different hospital departments. If a patient is dead, the porter transport the corpse to the chapel.
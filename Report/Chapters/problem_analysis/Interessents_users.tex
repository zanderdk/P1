%!TEX root = ../../Master.tex
\section{Individual stakeholders of indoor navigation in hospitals} % (fold)
\label{sec:interusers}

To understand the need for an indoor navigation system, we need to understand the different potential usersgroups and their needs for an optimized system. In the following segment we will try to bring forth the different motivations for wanting a more optimzied and personalized navigation system.

When it comes to indoor navigation in a hospital, there are three main groups we consider individual stakeholders, namely visitors, patients and staff. These groups all have different traits, that might be unique for a single group. However the groups we consider, only account for the traits of the reason behind the navigation, and not the personal traits of the navigator. So before we describe the differences of our three main groups, we must aknowledge that there are personal traits that might be shared across all of the groups.

People have natural differences in their ability to navigate. Some people can always tell you which way is north, some might lose their sense of direction if they spin in place, and many people have a hard time telling left from right, without any visual aid. They are also certain groups of people who are more prone to having difficulties navigating, such as the elderly, disabled people or people affected by mind altering substances. These groups of people also share a high frequency of visiting hospitals, and so it is increasingly important to adress their needs for navigational guidance.

What follows is a closer look at the different needs and traits of the three main groups.

\paragraph{Vistors} % (fold)
 \label{par:vistors}
 

An important trait that all types of vistors can share due to the nature of a hospital, is an elevated level of emotional distress. This is often due to the grief or otherwise sadness towards the illness of whom they are visiting. It can also be caused by other things, such as nosocomephobia (the fear of hospitals) which many people have a variating degree of. These traits can all be imparing, if one requires a focused mind when navigating, and it is important to take this into consideration when designing a navigation system. There are two types of visitors that we would like to focus on here, that being those who frequent hospitals and those who do not.

Visitors who do not frequent hospitals, and thus will not have any prior knowledge of the standard infratructure, will have a hard time navigating the hospital. These visitors will often not be used to navigate any large building, and therefore stationaire maps or other guidelines that intales a multitude of actions, which requires memorization, can be hard to adapt to. They will usually prefer asking members of the staff, or using the more simplified yet ineffecient forms of navigation, such as counting room numbers or being guided by a friend through a telephone. They will often need confirmation on their current whereabouts and the next course of action.

Visitors who do frequent hospitals, might still share some of the troubles of the non-frequent visitors, but they will however often know, in which quaters the patient they are visiting is situated, and how to get there. But unlike most vistors who do not frequent the hospital, these do not always prepare their visit in full detail, since they might be visting multiple times a week or even daily. An example of this, could be a parent of a hospitalized child, who decided to drop in during their lunchbreak. However it just so happened to be, that a part of a doctors schedule was freed up, and now the child is in another ward, recieving threatment an hour prior to the original schedule. Suddenly, a short visit turns into something much more, because the parent does not nessesarily know the details of the rescheduling. Having a loved one confined in a hospital is already an immense emotional stressfactor, but also having to deal with navigating to shifting whereabouts, can easily cause unnessesary agitation, which will likely be directed towards the doctors and nurses.

These are suboptimal conditions for a facility working with something as important as human lives, and staff should not be required to handle the stress of visitors, who are having trouble navigating.

%\paragraph{Medical Staff} % (fold)

%Just as with the visitors, Medical staff can be divided into two groups, those who frequent the hospital and those who do not. The members of the medical staff who frequent the hospital, will know their way around, at least they will know the part that they are working in, but even then, there can occur navigation problems. 
%\fixme{[Kasper: kan ikke finde nogen kilde der omtaler 'medical staff's problemer med navigation. Hvis det her skal med, skal vi lave et interview. Alternativt kan vi også bare ligge vores fokus hos dem som er hårdest ramt (patienter)n]}

%The other group could contain surgeons with rare specialties, who are only brought in on special occasions, and therefore does not have the layout of the hospital memorized.

\paragraph{Patients} % (fold)

Patients who either live at the hospital for extended periods or patients who visit the hospital on a weekly or even daily basis, are considered frenquent patients.Emengency patients or patients with a non-recurring medical issue, will be considered non-frequent patients. The latter of the two groups, share a significant amount of traits with the normal visitors, but there are some traits that are more common for patients.

A lot of patients are, by the nature of being a patients, not completely well. Many patients will not have their ability to navigate impaired by their medical condition, but for other it can be greatly reduced, sometimes to the point of non-existence. This could include anything from phsyical injuries such as broken bones or other conditions that confines a patients his beds, or it could be cognitive disabilities either chronic or temporaily medicinally induced. Other disabilities such as visual impairment and dyslexia will often render a visually based guidance system, such as maps, signs and arrows, to be of no virtual help.


\paragraph{Solving the problems}

Solving the problem with indoor navigation will require a possibility to adapt to a multitude of different needs, both when generating a route and when relaying the information to the user. Traditional methods of navigating, such as finding the shortest distance or the fastest path, might not be the best way to generate the route, since special needs of certain users can cause these methods to be counterproductive.


% section Interessents - Users (end)
%!TEX root = ../../Master.tex
\section{Interessents - Users} % (fold)
\label{sec:interusers}

\fixme{Aspekter som mangler at blive uddybet: Handicappede, ældre, folk med dårlig retningssans (såsom venstre-/højreblind), }


There is a broad spectrum of users in hospitals alone, who could benefit a lot from an improved indoor navigation system, the majority of which can be be divided into multiple subgroups:

\paragraph{Vistors} % (fold)
 \label{par:vistors}
 

An important trait that all types of vistors can share due to the nature of a hospital, is an elevated level of emotional distress. This is often due to the grief or otherwise sadness towards the illness of whom they are visiting. A lot of people also have a variating degree of nosocomephobia (fear of hospitals), or just feels generally uncomfortable in such places. These traits can all be imparing, if one requires a focused mind when navigating. There are two types of visitors that we would like to focus on here, that being those who frequent hospitals and those who do not.

Visitors who do not frequent hospitals, and thus will not have any prior knowledge of the standard infratructure, will have a hard time navigating there. These visitors will often not be used to navigate any large building, and therefore stationaire maps or guidelines that intales a multitude of actions, which both requires memorization, can be hard to adapt to. They will usually prefer asking members of the staff, or using the more simplified yet ineffecient forms of navigation, such as counting room numbers or being guided by a friend through a telephone. They will often need confirmation on their current whereabouts and the next course of action.

Visitors who do frequent hospitals, might still share some of the troubles of the non-frequent visitors, but they will however often know, in which quaters the patient they are visiting is situated, and how to get there. But unlike most vistors who do not frequent the hospital, these do not always prepare their visit in full detail, since they might be visting multiple times a week or even daily. An example of this, could be a parent of a hospitalized child, who decided to drop in during their lunchbreak. However it just so happened to be, that a part of a doctors schedule was freed up, and now the child is in another ward, recieving threatment an hour prior to the original schedule. Suddenly, a short visit turns into something much more, because the parent does not nessesarily know the details of the rescheduling. Having a loved one confined in a hospital is already an immense emotional stressfactor, but also having to deal with navigating to shifting whereabouts, can easily cause unnessesary agitation, which will likely be directed towards the doctors and nurses.

These are suboptimal conditions for a facility working with something as important as human lives, and staff should not be required to handle the stress of visitors, who are having trouble navigating.

\paragraph{Staff} % (fold)

 
Just as with the visitors, staff can be divided into two groups, those who frequent the hospital and those who do not. Those who frequent the hospitals could be nurses and doctors, cleaners and caretakers, porters and lunchladies. These members of staff will know their way around the hospital, at least they will know the parts their are working in, but even then, there can occur navigation problems. \fixme{[Kasper: Jeg mangler info omkring nuværende tilstande. vender tilbage her til senere]}

The other group could contain surgeons with a rare specialty, maintenance staff and generally employees who do not have the layout of the hospital memorized.

\paragraph{Patients} % (fold)


As with the other groups, patients can be divived into multiple groups. 

\chapter{chapter} % (fold)
\label{cha:test}
\section{section} % (fold)
\label{sec:section}
\subsection{subsection} % (fold)
\label{sub:subsection}
\subsubsection{subsubsection} % (fold)
\label{ssub:subsubsection}


\paragraph{paragrpa} % (fold)



% subsubsection subsubsection (end)

% subsection subsection (end)

% section section (end)

% chapter test (end)


% section Interessents - Users (end)
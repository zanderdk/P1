%!TEX root = ../../Master.tex
\section{Individual Stakeholders of Indoor Navigation in Hospitals} % (fold)
\label{sec:interusers}

\kanote{kilder skal indsættes. Mangler der kilder andre stedet end angivet?}

To understand the need for an indoor navigation system, one need to understand the different potential users groups and their needs for an optimized system. The following segment will try to bring forth the different motivations for wanting a more optimised and personalized navigation system.

When it comes to indoor navigation in a hospital, there are two main groups considered individual stakeholders, namely visitors and patients. These groups have different traits, that might be unique for a single group. However the groups we consider, will only account for the traits acquired while navigating, and not the personal traits of the navigator. So before we describe the differences of our two main groups, we must acknowledge that there are personal traits that might be shared across both of the groups.

Humans have natural differences in their ability to navigate. Some individuals can always tell which way is north, some might lose their sense of direction if they spin in place, and many have a hard time telling left from right, without any visual aid.[kilde5] There are also certain groups of individual who are more prone to having difficulties navigating, such as the elderly, disabled individual or individual affected by mind altering substances.[kilde6] These groups of individual also share a high frequency of visiting hospitals, and so it is increasingly important to address their needs for navigational guidance.[kilde1]

It is also important to consider the usage of software by the stakeholders. Many elderly are not accustomed with newer technology, and can often be distrustful and easily discouraged when having to interact with information technologies that they have not encountered before.[kilde8] Also when considering software usage, is the need for a reliable rate of information. Studies show, that most users will start to lose their patience after 5 seconds of waiting. [kilde7] These tedencies needs to be adressed, in order to service a greater number of stakeholders.

What follows is a closer look at the different needs and traits of the two main groups.

\subsection{Vistors} % (fold)
 \label{par:vistors}
 
An important trait that all types of visitors can share due to the nature of a hospital, is an elevated level of emotional distress. This is often due to the grief or otherwise sadness towards the illness of whom they are visiting. It can also be caused by other things, such as nosocomephobia (fear of hospitals) or claustrophobia (fear of confined spaces, such as elevators) which many individual have a variating degree of.[kilde2] These traits can all be impairing, if one requires a focused mind when navigating, and it is important to take this into consideration when designing a navigation system. There are two types of visitors that we would like to focus on here, that being those who frequent hospitals and those who do not.

Visitors who do not frequent hospitals, and thus will not have any prior knowledge of the standard infrastructure, will have a hard time navigating the hospital. These visitors will often not be used to navigate any large building, and therefore stationary maps or other guidelines that entails a multitude of actions, which requires memorization, can be hard to adapt to. They will usually prefer asking members of the staff, or using the more simplified yet inefficient forms of navigation, such as counting room numbers or being guided by a friend through a telephone. They will often need confirmation on their current whereabouts and the next course of action.[kilde2 and 5]

Visitors who do frequent hospitals, might still share some of the troubles of the non-frequent visitors, but they will however often know, in which quarters the patient they are visiting is situated, and how to get there. But unlike most visitors who do not frequent the hospital, these do not always prepare their visit in full detail, since they might be visiting multiple times a week or even daily. An example of this, could be a parent of a hospitalized child, who decided to drop in during their lunch break. However it just so happened to be, that a part of a doctor's schedule was freed up, and now the child is in another ward, receiving treatment an hour prior to the original schedule. Suddenly, a short visit turns into something much more, because the parent does not necessarily know the details of the rescheduling. Having a loved one confined in a hospital can already be an immense emotional stress factor, but also having to deal with navigating to shifting whereabouts, can easily cause unnecessary agitation.

%\subsection{Medical Staff} % (fold)

%Just as with the visitors, Medical staff can be divided into two groups, those who frequent the hospital and those who do not. The members of the medical staff who frequent the hospital, will know their way around, at least they will know the part that they are working in, but even then, there can occur navigation problems. 
%\fixme{[Kasper: kan ikke finde nogen kilde der omtaler 'medical staff's problemer med navigation. Hvis det her skal med, skal vi lave et interview. Alternativt kan vi også bare ligge vores fokus hos dem som er hårdest ramt (patienter)n]}

%The other group could contain surgeons with rare specialties, who are only brought in on special occasions, and therefore does not have the layout of the hospital memorized.

\subsection{Patients} % (fold)

Patients who either live at the hospital for extended periods or patients who visit the hospital on a weekly or even daily basis, are considered frequent patients. Emergency patients or patients with a non-recurring medical issue, will be considered non-frequent patients. The latter of the two groups, share a significant amount of traits with the normal visitors, but there are some traits that are more common for patients.

A lot of patients are, by the nature of being a patients, not completely well. Many patients will not have their ability to navigate impaired by their medical condition, but for other it can be greatly reduced, sometimes to the point of non-existence. This could include anything from physical injuries such as broken bones or other conditions that confines a patients his beds, or it could be cognitive disabilities either chronic or temporarily medicinally induced. Other disabilities such as visual impairment and dyslexia will often render a visually based guidance system, such as maps, signs and arrows, to be of no virtual help.[kilde4]


\subsection{Summary}

These are suboptimal conditions for a facility working with something as important as human lives, and niether medical nor other staff should be required to handle the stress of visitors and paitents, who are having trouble navigating.


% section Interessents - Users (end)
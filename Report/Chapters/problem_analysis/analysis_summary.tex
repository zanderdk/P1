%!TEX root = ../../Master.tex
\section{Analysis Summary} \label{analsum}
Before leaving the analysis of the initial problem, a summarisation will be given. In analogue navigation it was discovered that technologies currently in use require a lot of user interaction. Many of the methods of navigation, does not take different individual needs into consideration. They are either in fixed locations or in other ways unreliable. This can result in an inflation of the severity of a navigation problem, that occur while navigating or otherwise not in the vicinity of assistance. 

The stakeholders have a wide range of needs, both due to the traits imposed by being at a hospital and also due to the diversity of the individuals who visits or are admitted. It is paramount that a solution considers the frequency of disabilities in stakeholders, in order to be considered optimal.

It is of great importance to consider the requirements the hospitals have. One could design a great solution for the users of the navigation system, but if the solution is not fit for the needs of the hospital, the hospital will never implement the navigation system.

In regards to the analysed technologies, certain ones are favourable. Their ability to contribute as tools in solving core aspects of navigation and to adapt different user needs, makes them ideal choices for use in a optimal solution.
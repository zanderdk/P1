%!TEX root = ../../Master.tex
\section{Problems tied to the systems} % (fold)
\label{sec:Problems_tied_to_the_systems}

\paragraph{Phone}
The limitations regarding the phone, are high as the informer is limited by only having his/her voice as their tool. Miscommunication can occur as directions only can be delivered by words. As said before, this form of navigation strictly depends on an assigned personal to answer the phone. If no one is at the phone, it becomes utterly useless.
If the service is used often, more than one employee might be assigned to the phone. It could become an expensive service if there is a dedicated staff assigned to the phone. 

\paragraph{Signs}
The signs can be hard to spot if the  visitor is not familiar with the hospital, also a sign can be obstructed by other signs or if the room/hallway is filled with people. Another problem regarding the signs, is that they might be hard to use if the visitor have reading problems. If the one reading the sign cannot understand the language or is an illiterate, it becomes hard to find the information that is relevant. Too much information can also be displayed on signs so that it becomes confusing or hard to see the system behind the sign placements.
Also, if the visitor is in building A, and wants to visit a patient in building B floor 6. It could become a challenge to set up enough signs to guide the visitor to the right building and floor, without drowning the other visitors with non relevant information.

\paragraph{Maps}
A big problem with maps is they can become very complicated to get a overview of if they cover multiple floors \cite{map_confusing}. If the visitor is already inside the building, it can sometimes be difficult to figure out where they are corresponding to the map. If they are standing in a hallway it can be difficult to distinguish it from the others. 

\paragraph{The receptionist}
If the visitor arrives at an entrance different from the main one, they might not know where the receptionist is if they need help. The information received from the receptions have to be memorized when the visitor ventures away from the desk. This means that directions could become hard to remember if they have to get to distant location inside the building. A way the receptionist could help the visitor remember, would be to write a note but even so the text could be misunderstood or in other ways mixed up.

\paragraph{Colour coding}
If some rooms switches their function the stripes on the wall have to be repainted which will be a lot of work. If there are stripes leading to all the different departments, the information could potentially clutch up and become confusing. This method also shares a downside together with the signs, as people with reading disorder could have trouble with this form of navigation.

\paragraph{Porter}
The porter has other tasks he/she needs to attend to and will not always be available. Also, if the visitor have used the porter to get to a certain room, and wants to leave an hour later, they might not know where to find a new one. 

% section Problems_tied_to_the_systems(end)
%!TEX root = ../../Master.tex
\section{Problems tied to the systems} % (fold)
\label{sec:Problems_tied_to_the_systems}

\paragraph{Phone}
The limitations regarding the phone, are high as the informer is limited by only having his/her voice as their tool. Miscommunication can occur as directions only can be delivered by words. As said before, this form for navigation strictly depends on a assigned personal to answer the phone. If no one is at the phone, it becomes utterly useless.
If the service is used often, more than one employee might be assigned to the phone. It could become an expensive service if there is a dedicated staff assigned to the phone. 

\paragraph{Signs}
The signs can be hard to spot if the  visitor is not familiar with the hospital, also a sign can be obstructed be other signs or if the room/hallway is filled with people. Another problem regarding the signs, is that might be hard to use if the visitor have reading problems. If the one who reads the sign can't understand the language or is an illiterate, it becomes hard to find the information that is relevant. To much information cn also be displayed on signs so that it becomes confusing or hard to see the system behind the sign placements.
Also, if the visitor is in building A, and wants to visit a patient in building B floor 6. It could become a challenge to set up enough signs to guide the visitor to the right building and floor, without drowning the other visitors with none relevant information.

\paragraph{Maps}
A big problem with maps is they can become very complicated to get a overview off if they cover multiple floors\cite{map_confusing}. If the visitor is already inside the building, t can sometimes be difficult to figure out where they are corresponding to the map. If they are standing in a hallway it can be difficult to distinguish it from the others. 

\paragraph{The receptionist}

The receptionist can answer questions from the visitors. Very much like the phone, but with some key differences. The phone is more accessible as the one calling doesn't need to be at the hospital. The receptions has an advantage of being more precise. There will be less confusion as body language can be used in the answering of the question.

\paragraph{Colour coding}
Coloured stripes across the wall or floor that leads to the different areas of the hospital. In the entrance hall the visitors will see a wall with some lines of text with a coloured stripe behind it. One line might say "recovery" and if followed, will lead to the recovery department. Some departments also have an entire theme in a certain colour. In this way, it might become easier for some people to navigate the next time they visit, if they can remember the colours representing the department. 

\paragraph{Porter}
The porter helps patients and visitors get around at the hospital. The porter helps visitors find available beds if they have to stay at the hospital overnight \cite{ugd_port}. 

% section Problems_tied_to_the_systems(end)
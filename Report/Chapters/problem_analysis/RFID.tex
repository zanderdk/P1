RFID (Radio Frequency Identification) a radio technology that makes it possible, just like barcodes, to identify different objects. the main different between these two technologies is that the bar code is a line of sight technology and RFID is wireless, as long as it is within visible distance. The readable distance varies depending on the type of RFID tag. The information exchanged can be in many different formats such as text, numbers , audio or video, It depends on the size of RFID the tag built-in memory. /site[RFID_TEC1]

There are, different RFID tags they can be passive, active or battery-assisted. tags can be read-only "one times read" or read-write "multiply times", have a seial number or can be plank for the owner to write somthing on it. To put into perspective this to our subject, a read-only tag can be used to find an object such. a printer that does not change it's name or property. read-write tag, can be used to track patients, the information can be chance name and CPR number to match the person who wears it. /site[RFID_TEC2] /site[RFID_TEC3] 

http://www.scienceprog.com/how-does-rfid-tag-technology-works/

http://www.enigmatic-consulting.com/Communications_articles/RFID/Link_budgets.html

http://www.rf-id.com/6information/rfid_info_techexplained.htm

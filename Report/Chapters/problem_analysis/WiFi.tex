\subsection{Wi-Fi} \label{wifitech}
Wi-Fi is  technology that uses radio waves to transfer data between electronic devices. Smartphones, computers, tablets and headsets are devices that may have Wi-Fi together with many other electronic devices\cite{wifi_devices}.

Wi-Fi can connect to the internet via a wireless access point such as a router. Wi-Fi allows places that normally would not have internet connection, like sheds or gardens, to connect. This is possible as the router or wireless access point will allow the device to get a connection.

This is widely used as they serve as an convenient way of getting internet access. They are fairly easy to set up and used widely in the private homes and also used at a greater scale such as an airport\cite{wifi_works}. It is an convenient way of getting internet access as the user is in no need of getting a cable that physically have to be connected to a router.

It is relevant for this project as there have been invested in installing Wi-Di hotspots at the Danish hospitals\cite{wifi_hospi}. This means it would be possible to set up a service that will connect trough these hotspots.

A problem with Wi-Fi is how, as by Danish low, the hospital will have to log data from the users who connect to it\cite{wifi_log}. in order to only log the users who are using he Wi-Fi, a password would be optimal as it will prevent people bypassing he complex from automatically connecting. The range is also a limitations as typical routers will support a range of 46 meters indoor\cite{wifi_range}. Walls will reduce the strength of the signal, which means some areas od the complex might no have a good enough connection\cite{wifi_wall}. Handheld devices consumes more power when Wi-Fi is active, and quickly drains the battery as data is send and received\cite{wifi_batt}. The amount of power used is reduces as the device goes into a low power mode, which is activated once it has no more data to send. It goes into hight power mode when it is sending/receiving data and stays there for the duration of which it is sending/receiving the data.
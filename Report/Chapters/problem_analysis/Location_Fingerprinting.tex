\subsection{Location Fingerprinting}
In location fingerprinting, there are to main tecniques that have been choosen for this project. It is "QR-code" and "RFID".
Location fingerprinting is method of recognizing devices by giving them a electronic fingerprint, such as a unique id.

\subsubsection{QR-code} % (fold)


QR-codes are barcode that returns data when scanned. The code is represented by black boxes on a white square grid background. All by how the boxes are located, the code will send different data.

In order to read the data, a scanner have to be used. A camera from a smartphone can be used for this purpose, as long there are a app that supports format. Such apps can be found on nearly all smartphones \cite{QR_smart}. The QR-code can be used to lead users to websites and are used often in different kind of commercial ways\cite{QR_url}.

It easy to read the QR-code if you already have an app installed on your smartphone. It's also easy to make your own codes as Google have released a free tool to generate them\cite{QR_Google}.It's also rather easy to set up a QR-code system in a effective way\cite{QR_easy}.

The relevans for this project is how it will take little time to set up\cite{QR_rel1} and how some people allready knows how use them as they are widely spread\cite{QR_spread}. Even if a code is damaged it will still be usable, even if up to 30\% of the code is missing\cite{QR_dama}. This also means it will be possible to implant images in the code and still have it working\cite{QR_image}.

A reason for not to use the codes could be that some users might exploit the system. If the print their own codes with links certain websites it could be a very unpleasent experience for the other users\cite{QR_urlbad}. In worst case scenarios they could be used to steal personal information\cite{QR_information}.
In order for the codes to be efficient, they also have to placed at many locations at the hospital. The users should always be able to find them when they want to use them. This could leave to frustration if seems to be nowhere to find for the user.


\subsubsection{RFID}


what is RFID, RFID (Radio Frequency Identification) a radio technology that makes it possible, just like barcodes, to identify different objects. the main different between these two technologies is that the bar code is a line of sight technology and RFID is wireless, as long as it is within visible distance. The readable distance varies depending on the type of RFID tag. The information exchanged can be in many different formats such as text, numbers , audio or video, It depends on the size of RFID the tag built-in memory. 

There are, different RFID tags they can be passive, active or battery-assisted. tags can be read-only "one times read" or read-write "multiply times", have a seial number or can be plank for the owner to write somthing on it. To put into perspective this to our subject, a read-only tag can be used to find an object such. a printer that does not change it's name or property. read-write tag, can be used to track patients, the information can be chance name and CPR number to match the person who wears it.  

%!TEX root = ../Master.tex
\chapter{Problem Statement}

%Opsummering af problemanalyse. delkonklusion af prob anal

%Delformuleringer op til 3. Spørgsmål man kan svare på. Formuler spørgsmål ud fra konklusionen fra prob anal.


In the problem analysis we covered the different problems associated with indoor navigation in a hospital.
We discovered that there are many problems tied to the use of analogue navigation systems. We now have knowledge about the incompatibility with indoor navigations, and existing technologies such as GPS.
% Based on the problem analysis (\cref{cha:problem_analysis}), we have delimited the initiating problem (\cref{sub:init}) to one problem statement:

% \textit{\textbf{How can a software solution create an individually optimized indoor navigation path, through the use of an established positioning system and a portable computing device?}}


With this knowledge we constructed a list of specifications for a possible solution.


%!TEX root = ../../Master.tex
\section{Specification} % (fold)
\label{sec:specification}

What follows is the list of requirements that needs to be met, before a solution can be considered complete. These specifications are based on discoveries made in the analysis of the initial problem. Each requirement will be followed by a short description of why it is necessary for a complete solution, and references to which parts of the analysis that supports it.

\subsection{Usage Requirements}

Usage requirements are needs that must be met, before the solution can be considered to be meeting any and all anticipated users basic expectations for the interaction with the solution. This does not include any underlying computations, but only the ways of giving input to and representing those computations, in compliance with the needs of the using stakeholders described in \cref{sec:interusers}.

\begin{itemize}
	\item The user must have the possibility, but not be required, to personalize the generated route through the following parameters:
			\begin{itemize}
				\item Are stairs to be avoided or preferred?
				\item Are elevators to be avoided or preferred?
				\item Are disabled-friendly routes a must?
				\item Should the route be fastest in time, shortest in distance, or the least physical strenuous?
			\end{itemize}
\end{itemize}

In order to optimize navigation, the limitations of systems in current use must be considered. See \cref{sec:anal_nav}. A system that does not take personalized input from a user, will not be able to consider the needs for that particular user. As described in \cref{sec:interusers} these needs can differs immensely across user groups, and it is therefore important to be able to personalize the generated route.

The first three parameters are able to establish most boundaries the users might have, whether they are physical or psychological, while still keeping the amount of parameters to a minimum for simplicity. This is another requirement which we will expand on later. The last parameter is less essential, since all user groups could always use the least physical strenuous, however in order to optimize the route specifically for the user, this option should be given as well. This last parameter could also be an indirect input, based on other input, such as the travelling speed of the user and previous choices. In any way, the program should be able to choose different criteria for the choice of optimal route.

\begin{itemize}
	\item The user interface must be intuitive, simple and present the user with a overview of data representation.
\end{itemize}

It is essential to display all the necessary information, and nothing more. However, if the information is displayed statically, this can prove to be impossible, since the criteria for necessary information changes with the user. See both \cref{sec:anal_nav} and \cref{sec:interusers}. The advantage of a software solution in this regard, is the ability to shift the information displayed, based on the user's current situation. In order to fully optimize, this must be incorporated into a solution.

\begin{itemize}
	\item The maximum wait caused by route generation must not exceed 2 seconds.\label{irm_tid}
\end{itemize}

Studies have shown that the patience of a user, is fairly low. See \cref{sec:interusers}. The user must experience that the solution is better than the alternatives, before agreeing with that statement. It therefore become important to relay information before the user feels that time is being used inefficiently.

\begin{itemize}
	\item The program must represent the route instructions either as text, audio or pictographic.
\end{itemize}

Since the solution will be dealing with a wide variety of users, seen in \cref{sec:interusers}, the means of relaying information to the user, should cater to the different needs of the users.


\subsection{Technical requirements}

Technical requirements are all the requirements that will only indirectly affect the user. This concerns matters on resolving navigation problems while under certain limitations, such as being indoor causing momentary signal loss. These requirements are the backbone of the solution, all are thus paramount for the uttermost basic functions of a solution.

\begin{itemize}
	\item The program must be able to generate the optimal route to a given destination.
\end{itemize}

Problems with current navigation systems, include relying on the users ability to figure out the optimal route with sometimes confusing information displayed by a stationary map. See \cref{sub:map}. In a software solution this could, and should, be handled by the system.

\begin{itemize}
	\item The program must be able to position the user through the use of a wireless transmitting technology.
\end{itemize}

The ability to position the user, is the first step required in order to establish a route to any desired location. The position should be done wirelessly, since it is already an integrated part of the infrastructure, and would allow for optimal mobility. See \cref{sec:organization}

\begin{itemize}
	\item The user's movements must be traced even without communicating with external factors.
\end{itemize}

Because of the indoor environment present at a hospital, the user might momentarily lose the connection to a given positioning system. Therefore any device in charge of positioning the user, should be able to do this to a certain extend, without communicating with external factors. See \cref{wifitech} and \cref{subs_acel}.

\begin{itemize}
	\item The program must be able to detect deviation from the route, and recalculate accordingly.
\end{itemize}

This requirement originates from a need to comply with the basics of navigation. Since we want to rely as little as possible on the users abilities, we want the solution to be able to handle this as well. See \cref{intro}.

\begin{itemize}
	\item The program must be able to support different building complexes.
\end{itemize}

Any kind of tool takes a bit of effort to understand at first, and this solution will be no different. If the solution however was to be applied in multiple places, chances are that the user might already have had dealings with it. This could reducing the users need to familiarize with the solution, and will create a more pleasant user experience. See \cref{sec:interusers}.

% section specification (end)

This specification outlines a solution to the initial problem. In order to solve this solution it has to be split up into several subproblems.

\begin{itemize}
	\item How can a device position a user inside a multi layered building where GPS is not accesible?
	\item How can a software solution compute a route inside a multi layered building suiting a users preferences?
	\item How can a route be represented to a potentially disabled user in an intuitive way that still allows for a quick overview.
\end{itemize}

In this paper we will design a solution that fits the \fordinal{2} subproblem. This leaves us with the following problem statement:

\begin{displayquote}
    \textit{How can a software solution compute a route inside a multi layered building suiting a users preferences?}\label{sub:problem_statement}
\end{displayquote}

\sinote{Beskriv hvordan deployment fungerer. Beskriv modellering af data, krav af servere osv.}
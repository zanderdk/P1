%!TEX root = ../Master.tex
\chapter{Introduction}\label{intro}

In todays modern world, effeciency and optimization are more important than ever before. In all matters we strive towards having the best solutions possible, constantly employing the newest technological advances in the pursuit of a better world. With the current speed of technological development, every part of our society is evolving and techniques that once were considered state of the art, are quickly becoming outdated.

Not many years ago, this was the case with navigation. The GPS was introduced to the market, and road maps were suddenly outdated technology. However, this way of using software to automatically navigate you, has yet to be properly implemented into an indoor environtment.

Hospitals is an indoor environment, that is always trying to be on the frontier of technological advances. This is obvious, due to the fact that they are dealing with the preservation of human life, which is by many deemed to be our most precious belonging as a society. This puts a lot of pressure on the medical staff, to always achieve the best results, and work as efficiently as possible. For this to happen, the entire hospital must work as a combined unit, with as few unforseen tasks as possible.

A large amount of time is, however, spent by hospital staff doing trivial tasks, such as guiding patients and visitors around the hospital. \cite{findvejintro} \cite{timewaste}. This puts currently employed technology in a position of inadequacy when dealing with those tasks. And so it begs the question; how can this be improved upon, in the manner that GPS improved upon outdoor navigation?

This report will take a look into the possibilities of improving the way we use technology for indoor navigation, with the aim of answering the question:

\textit{How can a software solution optimize indoor navigation in Danish hospitals for visitors and patients with different prerequisites?}\label{sub:init}





% Do you remember the days, when you had to go on a road trip and printed out written instructions, or had to use a map to path your way? Well we barely do. We all use GPS navigation and have no intention of going back to the old system. But even though it has been so many years since we stopped using manual wayfinding in cars, we still have not figured out an efficient way to incorporate computers into indoor navigation. Why does the navigation assistance stop at the entrance, even though the hardest part of finding your way, might just start at the exact same moment? We feel this problem needs to be addressed, especially in streamlined facilities such as hospitals, where the pressure from society to optimize is constantly increasing.

% %\{How can indoor navigation in hospitals be optimized by generating a specific route for individuals with different prerequisites?}
% Our \textbf{initiating problem} is:

% \begin{displayquote}
%     \textit{How can a software solution optimize indoor navigation in Danish hospitals for visitors and patients with different prerequisites?}\label{sub:init}
% \end{displayquote}

% We formulated our initiating problem on the basis of the research we conducted on the subject, "Indoor Navigation in Hospitals". We choose this initiating problem because, todays hospitals or generaly buildings are getting bigger and more complex, this is a issue, and in the following chapters we will clarify and analyse the aspects relevant to our prolem. This analyse will cover socaial relevans, todays hospitals, the techloligy about navigation, and Theory behind it. and in the end, our solutions to some of the problem with navigation. 
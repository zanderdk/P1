%!TEX root = ../Master.tex
\chapter{Discussion}

\section{Possible program improvements}

Consider the following situation; an user wants to find the fastest route from first floor to third floor. The user does not have any disabilities so both elevators and stairs can be part of the fastest route. The current program would most likely guide the user to the elevator from the first floor to the third floor. However it would be a problem if the elevator was already in use at the sixth floor. The user would then have to wait for the elevator to descent. This could take some time. If the program had the ability to track the location of the elevator, it could use this information in the algorithm of the program. The program would then have enough data to guide the user up the stairs, providing the user with a faster route.

\subsection{Implementation}

In order to have the final program, several sections of it have to be implemented. The program will need to receive positioning data and use it when calculating the route, as of now the user have to manually type in in where he/she is. The program needs a better method of communicating with the user, as of right now the program prints the vertices the route is split into. It would be more optimal if the user interaction where with more detailed text as "Go right at the next corridor". The user communication does not necessary have to go trough a text interface, a graphical interface similar to the one of GPS's would also be a good solution\cref{fig:TomTom}. The interface should as previous mentioned only be similar as some of the represented data on GPS will be unnecessary for users. This is information like speed and driving related information as parking lots etc. A graphical interface similar to TomTom would help the users to navigate, as visuals will give a better overview of the complex and the route. 

\begin{figure}
\centering
  \begin{minipage}{0.45\textwidth}
    \centering
    \includegraphics[width=\textwidth]{tomtom_go_720.jpg}
    \caption{A TomTom GPS. \cite{diss_tomtom}} \label{fig:TomTom}
  \end{minipage}
\end{figure}


\section{Suggestions for Further Research}

The Flawless algorithm is made to work in complexes. So far hospitals have been the focus of this project, but other complexes should be able to integrate this system as well. Such complexes could be shopping malls, ferries or exhibition halls. 

The shopping mall could benefit from this system as wheelchairs would be able to find elevators with more ease, and customers in general would easier find the various hops they are looking for. Ferries could also benefit from the system as they have travelling businesses men/women who might need to work doing the their travel. Because of the stress that comes with travelling\cite{future_stress} it would make a more pleasant experience if some stress factors like navigating or getting lost could be removed or subdued. The current system would not work as it is at a ferry because of the way it gets the user position. The accelerometer that keeps track of how the user shift direction and moves trough the complex, would be disturbed every time the ship tilted or changed direction. A solution to this would be to solely rely on the Wi-Fi to find the users start position and to keep track of them when they are on the move. Exhibition halls are relevant to look at as s lot of people who have never been in these halls and could need help finding their way to the various booths. If the visitors could use the system to find their way to the booths they are looking for, less people would get lost and here would be less clutters of people as the times used on going from one pace to another will be reduced. 
\newline 
The program will be more useful for the visual impaired if it was able to give directions trough sound. This could be done by implementing Text to Speech\cite{diss_tss} and using it to read the directions the program outputs.

\section{External Factors}

A couple of observations that was made throughout the creation of this report considering the external factors, that can compromise the functionality of the program follows. The process of calibrating the graph in relation to the actual building, is considered to be paramount in order for the pathfinding algorithm to be able to find the optimal solution without failure. Since the algorithm is to remove all decision making from the user, it should never produce a less than optimal route. There also lies a problem, in the implementation of the solution. If the users are not properly informed about the solution, they will not adapt it due to its foreign nature, and will continue using the status quo methods of navigation.



Finde på en ny rute undervejs
oplæsning - future

Diskussion: synes den er noget tynd. Jeg mangler noget reflekterende. Hvad skal der til før at det bliver et system der kan bruges af jeres målgruppe. Hvilke eventuelle forbedringer til jeres løsninger er der og hvordan kunne disse implementeres?
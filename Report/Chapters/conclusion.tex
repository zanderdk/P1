%!TEX root = ../Master.tex
\chapter{Conclusion}

To conclude, the aim of this report was to analyse the need for and the requirements of an indoor navigation software system, which would allow users to personalize their navigation, in order to offer a better alternative when navigating a large building complex, such as a hospital. It was shown that for current navigation methods certain issues with reliability and individual needs will occur, and that certain user groups are not able to effectively utilize the current analogue systems. In order to alleviate these issues, newer technologies with relation to navigation was examined for insight into a way of solving the problems. The demands that needs to be met were defined and due to time constraint, a delimitation was made in order to focus the effort on a core of requirements, namely finding a way from A to B.

In order to solve the navigation problem, research into graph theory and relevant pathfinding algorithms was conducted. The strengths and weaknesses of different types of algorithms and the way of mapping and representing them was described, and the importance of calibrating the graph was discovered. 

Combining the knowledge of graph theory and indoor navigation issues, certain problems arose when applying graph theory to fit the requirements of navigation in hospitals. There were problems with how to map the building with graph theory, while conserving infrastructural information like hallways and stairs. These were solved through a modelling process that while it does not resemble a real building visually, preserves all the needed information, while storing it in a more compact manner. Another major issue was the matter of algorithm runtime while using traditional algorithms. The solution to this problem was a way of precalculating certain paths before runtime, and this FLAWLESS algorithm became the cornerstone for the entire solution.

The program that was developed in accordance with the results from the analyses, is able to generate a personalized route suiting the preferences of the user. The input for this generation is the current location of the user and a destination location. The program can be set to generate a route that avoid stairs, elevators or nothing at all, providing maximum accessibility for all users. The program outputs all the vertices the most optimal path consists of.
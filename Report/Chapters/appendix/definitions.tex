%!TEX root = ../../Master.tex
\section{Definitions} % (fold)
\label{sec:definitions}

\begin{description}
	\item[Navigation] The study of monitoring and controlling the movement of a craft or a vehicle \cite{wiki_navi}. In this paper we will broaden the definition by also including pedestrians.

	\item[Location Based Services] Location based services is a solution which includes location platform, applications and integration capabilities.
	
	\item[Positioning] The spatial location of an entity. This doesn't involve direction or speed \cite{wiki_pos}.
	
	\item[Indoor] A building where Global Positioning Systems (GPS) is not available.
	
	\item[Building levels] Different floors inside a building that are not connected to each other on the XY-axes other than via stairways, lifts etc.
	
	\item[Navigational instrument] A device that aids in positioning and navigation of an entity. This could be a compass, a map or a GPS enabled device.
	
	% \item[3D-building] A space defined by its XYZ-axes. It can contain 1 or more floors where each floor is in a certain range of the Z-axis. Each floor is often divided into rooms. Every room can be described as a subspace of the floor the room is situated in.
	
	\item[Spatial reference system] A spatial reference system (SRS) is a coordinate-based local, regional or global system used to locate geographical entities \cite{wiki_srs}.

	\item[Wayfinding]Finding a path from A to B.
	
	\item[Wi-Fi] A technology that uses radio waves to transfer data between electronic devices.
	
	\item[FLAWLESS] Fused Locale A* With Level-calculation Executed at System Startup

\end{description}

% section definitions (end)
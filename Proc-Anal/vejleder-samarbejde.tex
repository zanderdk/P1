\subsection{Samarbejde med vejlederen}
        % Har I haft en samarbejdsaftale med jeres vejleder? I så fald: har den fungeret tilfredsstillende? 
        % Hvordan forberedte I møder med jeres vejleder? 
        % Hvilken type vejledning ønskede I fra vejlederen? Hvilken type vejledning fik I?


Ved første møde med vejlederen, blev vejleder samarbejdsaftalen introduceret. Aftalen bestod af 7 punkter. Samarbejdsaftalen har fungeret fint som en base for samarbejdet mellem gruppen og vejlederen, så den kan fint siges at have fungeret tilfredsstillende. Møder med vejlederen blev typisk forberedt så sent som muligt, således at de seneste problemer/spørgsmål kunne komme med i dagsordenen. Forberedelsen foregik ved at en computer blev tilsluttet en projektor, hvorefter gruppens medlemmer kunne komme med input til dagsordenen. Dette kunne f.eks. være spørgsmål til vejlederen. Disse møder blev holdt med ca. en uges mellemrum, eller aftalt efter deadlines osv.

Den ønskede vejledning fra vejlederen var typisk spørgsmål omkring struktur, f.eks. hvad forskellige dele af rapporten skal indeholde. Ellers blev vejlederen brugt, specielt i rapportens afslutning, til at få respons på de arbejdsark der blev sendt minimum 24 timer før mødet.
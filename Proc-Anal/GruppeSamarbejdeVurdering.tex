\subsection{Gruppesamarbejde}
\paragraph{Samarbejdsaftale}
Finger reglen som var nedskrevet blev taget i brug indtil at gruppen var blevet bedre til ikke at afbryde hinanden og give mere plads. Gruppen var også gode til skrive referater under hvert vejlederemøde, hvilket var en opgave som folk selv meldte sig til. I starten af projektet var gruppen god til at aftale hvornår det var at næste møde skulle finde sted, i forhold til mødetid og hvornår man regnede med at tage hjem igen. Forsinkelser blev indberettet hvis man var forsinket, disse meddelelser kom indenfor 20 minutter, når man var sikker på at det ikke var muligt at møde op til den aftalte mødetid. Dog senere hen i projektet blev det mere sløvt med at melde sig forsinket hvis man ikke kunne komme til den aftalte tid.

\paragraph{Gruppetrivselen}
Gruppen har været god til at sørge for at der er en god stemning i lokalet og når der har været et arrangement. Ofte efter at dagens arbejde var over, har der været tid til et spil kort eller en film. Når der har været frokostpause, har gruppen sendt to mand ned for at hente brød og pålæg, for at kunne have fællesspisning i grupperummet bagefter. Alle betalte så deres andel af udgifterne ligegyldigt hvor meget man spiste eller ikke spiste. Dog har der været problemer med at ingen rydde op efter maden.

\paragraph{Lektier og arbejdsopgaver}
Der manglede ofte overblik over hvad der var lavet og hvad folk arbejde på. Det har så betydet at hvis nogen har skrevet i det samme dokument har man skulle bruge tid på flette teksterne sammen, hvilket har været spild af tid. Når der var udeligeret lektier til næste dag, var det heller ikke at alting blev lavet. Enten fordi at man ikke kunne finde tid til dem eller fordi man havde gabt over mere end man kunne sluge. Når man ville komme med kritik af et stykke arbejde har gruppen ikke været for nådefulde og derved ikke kommet med alt kritikken. 
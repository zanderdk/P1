Proc-anal:

Forventninger til P1

vejleder:

- Hvilke forvendriner havde vi til vores vejleder?
- Hvordan var vores arbejde med vejleder?
- Hvilke forventninger havde vore s vejleder til os?
- Hvad har været godt og skidt

Projectarbejde:

- Hvad var vores forventninger til P1 project arbejde? 
- Hvad skal vi ende ud med i P1?

Projectplanlægning:

- Brainstorm:
i starten af forløbet, lavede i barinstorm på tavlen, hvor vi delte det op i 3 store  emner, "Indoor navigation", "Problemer der skal håndteres" og "Intressenter". Disse emner blev delt ind i under emner, hvor vi smed relevante problemer og emner op omkring navigation eller SOTA som kan buges til navigation. 

- Planlægning:
Inden for den første uge af P1, lavede vi en tids plan for P1, tids blev lavet (ses her under) med poster så der let kunne laves om, eller flyttes rundt på emnerne, hvis vi havde brug for mere tid eller hvis et emne blev færdigt. Hen i slutningen af Projectet blev der lavet  en tidsplan v. 2.0. vores arbejde, er blevet udarbejdet  efter tidsplanen, for at kunne have et overblik over hvad der skulle laves og hvornår det skulle være færdigt.

- Emner:
- Materialer:

inddeling af arbejde og hvordan

- Hvordan arbejder  vi og hvorfor 
vores gruppe har primært arbejdet i små gruppere, fordi vi mener at for mange på en opgave ikke nødvendigvis giver bedre  resultat og da gruppen relmessig discutere om forskellige emner, har det været en god ide at lave så grupper så en discussion ikke har stoppet alt arbejdet, men alt har været snakket  igennem i gruppen, selvom vi har arbejdet i mindre grupper. 

- belbin rolle fordeling og grupper
I gruppen har vi under project arbejdde og materiale søgning arbejde i små grupperr, hvor vi har arbejdet ud fra belbin arbejds model, hvor vi har laveet  gruppen så mindst et medlem har stor viden inden for emnet eller programering, så denne person kunne forklarer og vise hvordan han har løst opgaven. 
Når vi har lavet grupperog fordelt arbejdet, har vi sat nogle emner op på det der skal laves og folk har kunne vælge dem de ville arbejde med og dermed delt os i små grupper, så vi har kunne lave flere ting på en gang, men hver dag har vi lavet gruppe møde på hvad der er blevet laver  og hvor langt de enkelte grupper er. Efter gruppe arbejde har vi snakket  om det i hele  gruppen, og givet  hinanden kontroktiv kritik.
 
- discussion
Når vi har valgt at arbejde i små grupper, har vi også været ude for at skulle discutere noget, discussion har startet i den lille gruppe er blevet  discuteret vidre i hele gruppen så alle er  kommet med input og de positive og negative sider og blevet vendt. 
 
- tegninger
Under mange af vores discussioner, har vi illustret det for gruppen på tavlen, så alle kunne være  med og her er et af ex. på vores discussion omkring håndtering af "Floors", der er blevet lavet mange gode ting på tavlen, men ikke alle ting er der blevet taget billeder af.

- Andet:
Under materialle læsning, sendte vi en gruppe ud på Syghus Nord Aalborg,  for at  tage billeder af hvordan navigeringen forgik, såsom farve kode, kort og skilte. 

- Hvad har været godt og skidt

overvejelser:

- hvad har gennerelt været godt og skidt
- hvad kan vi gøre bedre
- hvad kan vi bruge i P2
- Hvad kan bruges i P2
- Project  forslag til P2

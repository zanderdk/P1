\documentclass[article,11pt]{memoir}
\usepackage{fourier}
\usepackage[utf8]{inputenc}
\usepackage[danish]{babel}
\usepackage[T1]{fontenc}

\begin{document}
\section{Referat 10-12-2013 08:30}

\paragraph{Status}
Gruppen forklarer at der ikke er skrevet så meget til rapporten, men at fokus har lagt på at sørge for at de forskellige kapitler stadigvæk passer ind. 
De to hovedalgoritmer er lavet. Vejlederen forklarer at der ikke skal bruges for meget tid på programmet. Det er rapporten som gruppen bliver bedømt på, så alle de overvejelser der er lavet skal skrives ned. Hvis koden er god og den er meget gennemtænk, men ikke dokumenteret så kan det jo være "lige-meget".

\paragraph{problemformulering}
Der er 3 undersætninger som aldrig bruges og gruppen vil have af vide hvor vidt de er relevante. De 3 punkter er skrevet men referer ikke til nogle steder, de bliver dog afrundet længere nede. I stedet for at have dem som punkter, så bruge dem som en tekst binder imellem specifikationen. 
Den problemformulering som står tilbage er god nok. Gruppen nævner at fokuseret er flyttet mere over til at lave navigation i højder, hvor at fokus skulle være på de trapper og elevatorer. Vejlederen nævner at man eventuelt kunne afgrænse sig så det passer.
Kun et punktum efter en forkortelse: etc.. --> etc. 

\paragraph{Graf-teori}
Graf teori skal skrives til da det ikke er færdigt endnu. Mange af de ting som står kommer ikke til udtryk i rapporten. I afsnittet hvor at der forklares hvordan det skal moduleres, bør flyttes til efter at A* er forklaret. I graf-teori mangler der definition på "vægtet" eller "directed" eller "undirected". 
4.1.1 bør overvejes om det kan flyttes ned.
Illustrationerne til 4.1.1 er ikke fyldestgørende. Sproglige problemer og mange ting bliver ikke forklaret. Obstacle. Hvad er et valg? Forklar bedre hvad der sker på figurerne. 
Fig 4.3 "then A-algortihm expands the vertexes"
Lav et implementeringsafsnit som forklarer hvordan der gøres i dette projekt. Når der forklares om koordinatsystemet så referer tilbage, ellers kan de komme ud af kontekst.
4.2 Hvad er en pathfinder? Der forklares ikke hvad formålet er med denne algoritme. Der mangler mere teoretisk tekniske kapitler om hvordan det fungere og hvilke problemer der er.
Fig 4.4 der mangler en forklaring p hvad symbolerne og tallene betyder.
4.5, 4.7 det er svært at se sammenhængen  
Fig 4.6 bliver heller ikke forklaret. Det er ikke til at se hvad der vises.
Dijkstras måde at søge på bliver ikke forklaret godt nok. nabo-knuderne bliver kun nævnt en gang. Det ville være bedre hvis alle trinene bliver forklaret mere dybdegående. Vejlederen anbefaler at algoritmen deles op, og kommentere så man kan læse hvad der sker "funktionen bliver kaldet". 
Forklar hvad Admissible heuristic value bliver heler ikke forklaret.
Starten af side 36 skal forklares bedre.
I graf-teori mangler der også en forklaring på hvordan at en matrix kan repræsenteres på forskellige måder. 

\paragraph{Implementering}
Implementerings delen skal der forklares hvordan algoritmen bruges i C og hvad der er "oversat".
5.2.1 Det er ikke forklaret hvordan virker i forhold til rammene. "I C ville man gøre det med et array".
I kapitel 5 kunne der skrives hvilke krav som sættes til koden. Gruppen kunne godt tænke sig at lave et flowchart over hvordan prototyperne kaldes, det godkendes af vejlederen hvis det kan bruges. Hvordan skriver man pæn kode, og hvordan skal der sættes brackets?
Vejlederen anbefaler at der forklares hvordan de forskellige structs virer og hvad de indeholder. Der kan sagtens medtages tegninger.
gang med at vide det mere konkrete.
Vejlederen anbefaler at der er mange kodeeksempler i implementeringsafsnittet. Der er ofte en abstrakt forklaring på programmet hvorefter at man kan gå i 

\paragraph{Andet}
Vejlederen påpeger at dr skal bruges de samme betegnelser for hvad de forskellige ting heder. Hvis et term et sted hedder 'x' men et andet sted heder 'y' skaber det unødvendig forvirring.
I future works skriver man om hvad der skulle gøres for at få det til at virke i virkeligheden.
4.3 Flowchart skal nok flyttes ned tl løsningen, samt blive mere detaljeret.
Gruppen vi gerne vide hvordan at det skal forklares at al det abstrakte er nuller og ettere i koden. Gruppen skal vise at de ved hvordan det abstrakte fungere og hvordan det rent matematisk skal repræsenteres. Matrixerne skal beskrives for at forklarer hvordan det er at man kunne bruge dem.\newline

\paragraph{Mødet endte klokken 09:35 og næste møde aftales til mandag d. 16. december kl. 12:30}
\end{document}
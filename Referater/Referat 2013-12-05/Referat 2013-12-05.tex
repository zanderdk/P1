\documentclass[article,11pt]{memoir}
\usepackage{fourier}
\usepackage[utf8]{inputenc}
\usepackage[danish]{babel}
\usepackage[T1]{fontenc}

\begin{document}
\section{Referat af vejledermøde - 2013-10-31}
\paragraph{Proces Analyse}
Bivejleder nævner at proc. anal. mangler og at gruppen burde overveje at få samlet papirer og figurer til brug i proc. anal. Bivejleder foreslår at gruppen afsætter 15 min. om ugen til at gøre status på arbejdsprocessen.
\paragraph{Kommentarer til rapporten}
Bivejleder foreslår at bruge "software, hardware and infrastructure" i stedet for "software".\\
Bivejleder kommenterer at kravet til at følge ruten mangler eller skal laves på en anden måde.\\
Ligeledes påpeger bivejleder at der mangler begrundelse for den valgte afgrænsning.\\
Bivejleder ønsker genovervejelse af figur 4.3 eller en yderligere forklaring til figuren.\\
Bivejleder foreslår at figur 2.3 benyttes til at forklare Dijkstras algoritme, gruppen argumenterer for at den vil \\blive unødvendig kompliceret at kigge på, mens figur 2.3 på samme repræsenterer et meget simpelt problem.\\
Bivejleder påpeger at der indgår ikke i kravsspecifikationen hvor detaljeret modellen skal være. Gruppen er enige i at beskrivelsen mangler/skal flyttes.\\
Bivejleder påpeger at gruppen er nødt til at beskrive koordinatsystemet vi anvender.\\
Bivejleder foreslår at anvende en figur til at beskrive hvordan en bygning modelleres. Gruppen er enig.\\
Bivejleder ønsker at der skrives et organisationsafsnit som der i løsningen så kan afgrænses væk fra.
Simon spørger hvad bivejleder mener organisationsafsnittet skal indeholde.
Bivejleder forklarer at afsnittet skal beskrive den organisation hvor løsningen skal bruges. Med organisation forstås der en samling af personer og infrastruktur der tjener et bestemt formål.\\
Bivejleder kommenterer at B2-19 er "godt med " på organisationsafsnittet.\\
\textbf{Næste bivejledermøde aftalt torsdag 2013-12-12} 

\end{document}
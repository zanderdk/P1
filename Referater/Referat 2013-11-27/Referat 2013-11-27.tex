\documentclass[article,11pt]{memoir}
\usepackage{fourier}
\usepackage[utf8]{inputenc}
\usepackage[danish]{babel}
\usepackage[T1]{fontenc}

\begin{document}
\paragraph{Feedback på kravspecifikationen}
Der hvor kravspecifikationen bliver sat mangler der det perspektiverende, at det skal trækkes noget mere ud af analysen. Den skal være mere reflekterende. Det skal bedre vises hvorfor hvilke krav der er, så man kan hoppe ind i tankegangen. Der skal være flere begrundelser (50 mb)
Eventuelt skrive op med "description", at havde et formelt punkt og så noget tekst med hvordan og hvorledes.


\paragraph{Om censor}
Lagt vægt på at censor vil havde god kontekst, og at analysen hænger sammen med resten. Gruppen skal bedømmes på hvor gode de er på analysere sig frem til at program og løse det. At man kan ligge et ordentligt grundlag. Det vigtigste er helheden.


\paragraph{Output af program}
"Hvad forventes der når programmet skal give et output?"
Der forventes at der kan illustreres hvordan løsningen virker(det er fedt til fremlæggelsen)
(Man må godt komme med noget nyt til fremlæggelsen... måske)
At resten af løsningen ikke kan laves ud denne delløsningen skal reflekteres over og diskuteres.
Den retning der er valgt er vigtigere end at tage højde for de andre delproblemer. Nævn det i reflektionen at der blev opdaget problemer undervejs.


\paragraph{Initierende problem i indledning}
At kunne referere til et nært kapitel er helt okay. (Alexanders forslag til hvordan man kunne gribe indledningen an)


\paragraph{Politik i organisation}
Der er skrevet om hvordan uforudsigelige hændelser kan påvirke programmet. FEEDBACK - Det er godt at have med og passe godt ind
det som gruppen ikke føler er relevant men så kom med pga. dialog med vejlederen, bør måske fjernes. Hvis det virker malplaceret, så er det det også.
Eventuelt heller se på hvordan IT-afdelingen vil have det med løsningen eller regionen, hvis det virkelig skal med.


\paragraph{Tidsplan}
Alt er fjong


\paragraph{Problemstatement}
Feedback på hvordan denne del er sat op: Den som var udkommenteret ligger mere op til løsningen hvor at den brugte er mere opsummerende (?). tag det anede spørgsmål og lav det om til et ikke spørgsmål, men som en "problem setting". 
Kravspecifikationen ligger efter problemformuleringen.
ndrer navnet på kapitlet til "Problem Defination"
initierende problem --> problem analyse --> prob form --> krav --> udvælgelse til løsning(\textbf{scope of project})


\paragraph{Model og grafteori}
Der mangler noget om hvordan man modeller sygehuset. Kom frem til det i teoriafsnittet eventuelt.
"En bygning ser sådan her ud, og et sygehus ser sådan her ud med så mange trapper og så mange etager...."
At modellere den rigtige verden ved hjælp af grafer.
Vi skal beskrive hvorfor det er at gruppen bruger algoritmerne på den måde gruppen nu engang her gjort. At Astar har nogle begrænsninger så derfor har gruppen kommet op på en speciel løsning der virker her.
Starte teori ud med hvordan sådan en bygning ser ud og hvordan sådan en skal modelleres(og det er jo med grafteori).


\paragraph{Andet}
Reflektion --> konklusion --> perspektivering(future work)
Kildelisten skal være før appendix
Når et term bliver beskrevet må man godt hive det ud for sig selv og så få det beskrevet. Så læseren ved at gruppen har fattet det.
Sørg nu for at alt i teori er beskrevet! Også funktionerne som \textbf{distBetween}
Hvis der laves hjælpeprogrammer kan man godt nævne dem i rapporten, og den skal inkluderes på CD'en. Når man har lavet arbejdet så skal man da også nævne det og få credit for det.



Næste vejlederemøde blev lagt til tirsdag d.3 klokken 10:30
\end{document}
\documentclass[article,oneside,11pt]{memoir}
\usepackage{fourier}
\usepackage[utf8]{inputenc}
\usepackage[danish]{babel}
\usepackage[T1]{fontenc}

\begin{document}
\section{Referat af vejledermøde - 2013-10-31}
\paragraph{Sprog}
Vejleder fortæller at Kurt (semesterkordinator) har informeret os om at der skulle have været skrevet på dansk. \\
Hvis der skrives på engelsk kan man få en engelsk censor og eksamen kan komme til at foregå på engelsk.
\paragraph{Init. Prob.}
Vejleder mener at det initierende problem bevæger sig over i løsningsforslag (konflikt med bivejleder). \\
Vejleder mener ligeledes at det initierende problem skal drejes over på noget med digitalisering, da det initierende problem skal være løsningsorienteret i forhold til at vi studerer software. \\
Vejleder ønsker at initierende problem skal afgrænses mere, og vejleder ønsker ligeledes at initierende problem efterfølges af en general indledning der fortæller hvad vi ønsker at forklare i problem analysen.\\
\paragraph{Rapport indhold}
Simon spørger hvorvidt vi skal beskrive low-tech løsninger, vejleder bekræfter at det er en god ide at beskrive low-tech løsninger der ikke direkte relaterer sig til vores interesse.\\
Interessent afsnittet skal være mere behovsorienteret sådan at det bedre kan relateres til et produkt.\\
Vejleder anbefaler en afgrænsning af emner under førende teknologier.\\
SotA skal drejes over på platforme i stedet for algoritmer.\\
Vejleder ANBEFALER at vi kigger på lokations baserede services under positionering.\\
Vejleder siger at vi skal analysere den infrastruktur der er nødvendig for at positionere objekter indendørs. Herunder komme ind på om infrastrukturen er til stede.\\
Prob. Anal. er for bred. Skyldes til dels at vi går over i løsningsforslag. Vejleder foreslår at vi kun forklarer og analyserer ting der er direkte relevante for vores løsning.\\
\textbf{HUSK KONTEKST i forhold til vores problem.}\\
Under maps skal det specificeres at der er tale om analoge kort.\\
Under eksisterende systemer skal fordele inddrages på niveau med ulemper\\
\paragraph{Rapport teknisk}
Vejleder spørger indtil titelblad og fortæller at vi ikke må bruge visse dele af titelblad. \\
Vejleder forespørg at navn på pdf ændres til gruppenavn.\\
Vejleder ønsker at overskrifter gøres mere beskrivende og alle afsnit indledes med en indledning.\\
Vejleder ønsker en klarere adskillelse af antagelser og påstande underbygget af kilder, vejleder påpeger at der mangler MANGE kilder. Vejleder påpeger at vi skal være opmærksomme at underbygge påstande med kilder.\\
Vejleder anbefaler at emner under eksisting systems der alle relaterer til personkontakt slåes sammen. Ligeledes vil det gøre det muligt at generalisere visse metoder.\\
Vejleder anbefaler brug af flere og bedre figurer, da det bedre illustrerer det omtalte emne.\\
Vejleder kommer ind på at man måske burde lægge eksisterende systemer og problemer der relaterer sig til systemer sammen.\\
\textbf{Vejleder anbefaler at algoritmer og positionering flyttes til et teoriafsnit som efterfølger problemformuleringen.}\\
Teknologianalyse mangler.\\

\paragraph{Diverse}
Vejleder vil fremsende introduktion til LBS og artikel omkring indendørs navigation på CS.\\
Skift væk fra model fra PV2 til model fra PV 5-6 (Problem-baseret projektarbejde). Brug evt. lordon (staves?).\\
Til statusseminar skal der udarbejdes læsevejledning (se yderligere forklaring i fremsendt dokument).\\

\textbf{Næste vejledermøde onsdag 2013-11-06 kl. 09:15 i B2-6.}
\end{document}
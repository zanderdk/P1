\documentclass[article,11pt]{memoir}
\usepackage{fourier}
\usepackage[utf8]{inputenc}
\usepackage[danish]{babel}
\usepackage[T1]{fontenc}

\begin{document}
\paragraph{Kravspecifikation}
Kravspecifikationen skal laves så den skal tage forhold til det hele. Og så derefter afgrænse til det vi vil se på. "Det er de krav vi har til vores delløsning". i den indledede tekst kan man skrive om de problemer der er fundet og så lave en god overgang til kravene derefter. 


\paragraph{Problemformulering}
%Lav en opsummering på analysen, start med problemformuleringen. Et helt kapitel som hedder problem definition, hvor analysen bliver opsummeret. Så skriv noget á "hvordan kan man digitalt
Vi kan lave en bred problemformulering og så dele den op i delproblemer. Hvor man så afgrænser sig til en del af problemet. (delspørgsmål/delproblemer)
f.eks "hvordan modellere man etager?"  


\paragraph{positionering}
Det er ikke nødvendigt at skrive vildt meget om hvordan man positionere sig. Gruppen skal ikke bruge en masse tid og energi på at få det forklaret, dog er det helt fint lige at runde emnet. 
Det er vigitgt at gruppen for forklaret hvad en 'position' er og hvad der forstås med det.


\paragraph{Graf}
"Smartcampus" 
Der er også steder som patienter og besøgende ikke må komme ind. Gruppen behøves ikke at modelere bygningen, det er fint bare at skrive nogle andre har gjort det, at lave en afgræsning.


\paragraph{Introduktion}
Introduktionen kan bruges som en indledning til en indledning. Der skal også være en reel hård indledning. Der blev nævnet om social relevans kunne bruges som den rigtige indledning. Det skal være en forret til resten af rapporten, hvorfor er der brug for denne løsning? En indledning kan godt ligge på lidt over en side. Man skal kunne læse indledningen og konklusionen og så få et overblik over rapporten.  


\paragraph{Organisation}
Infrastrukturen skal være på plads når det er at produktet skal bruges. Så hvad kræves der? Det er relevant nok at der bliver beskrevet hvordan beslutningsprocessen finder sted. Målgruppen er jo sådan set sygehuset og hvor end-brugeren er de besøgende. 


\paragraph{Sygehus Nord}
Der hvor der bliver skrevet at gruppen har været derude, bør der "koges suppe på det". 
Gruppen var i tvivl om hvordan det skulle bruges senre. Når løsningen skal laves kan man skrive at der tages et konkret udgang i sygehus nord, der er så mange indgange og så mange elevatorer osv. "Vi har et godt kendskab til sygehuse nord og derfor vil vi også tage udgangspunkt i det".
Beskrive hvad formålet var i at besøge sygehus nord.


\paragraph{Initierende problem}
Om man kan afgrænse os fra staff i starten. Okay hvis der kan findes en super god grund. Begrundelsen pt. er at det er svært at finde kilder på at staff har problemer. Man kan sige at den primære bruger som har brug for sådan et system vil være dem som kommer udenfor, der er dog stadigvæk tvivl om det er en acceptabel påstand. 


\paragraph{Algoritmer}
Der mangler et afsnit omkring grafteori generelt. Der ska beskrives ting SSP. Først at beskrive dijkstra algoritme og så bygge ovenpå med Astar er en god måde at skrive det på. Hold læseren i hånden. (plug ind til LaTeX med at man nemt kan skrive algoritmer ind). Når algoritmen beskrives kan vi dele den op i mindre dele, og beskrive de separate dele.  


\paragraph{Bivejleder}
Brug bivejlederen til at få nogle andre øjne på det som skrives.


\paragraph{Andet}
skrive om delproblemerne i forhold til hvordan det skal virke med positioneringen og den data der modtages. Samt også hvordan bruger kommunikationen skal forgå. "sygehuset kunne eventuelt lave nogle andre hjælpemidler, såsom streger i gulve programmet kunne referer til", få det med i analysen og derved bruge nogle af de ting som der allerede er skrevet om. 
Teorien skal relatere til problemet. "Vi skal finde vej fra 'A' til 'B' så derfor ville vi undersøge dette..."



Næste vejledermøde blev aftalt til onsdag d. 27 klokken 11:00 
\end{document}
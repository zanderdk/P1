\documentclass[article,11pt]{memoir}
\usepackage{fourier}
\usepackage[utf8]{inputenc}
\usepackage[danish]{babel}
\usepackage[T1]{fontenc}

\begin{document}
\section*{Læsevejledning}


Under læsning af rapporten, bør der ses bort fra manglende kilder.
Der ønskes heller ikke respons på layoutet da det ikke er endeligt som det ser ud nu. 
Der må gerne kommenteres på vores definitioner. 
\\*

Introduction \\*
Vi søger respons på tonen, men ønsker ikke at få vejledning i forhold til længden det det indholdsmæssige.
\\*

Problem analysis\\*
Vi ønsker ikke respons på "Initiating problem" da vi ikke har tænkt os at ændrer den, men tager genre imod kommentarer. 
\\*

Social relevance \\*
Dette afsnit er ikke færdigt, men vi vil gerne have respons på indholdet. 
\\*

Analogue navigation \\*
Her vil kommentarer på indholdet være velkommen, og afsnittet anses for at være tilnærmelsesvist færdigt.
\\*

Stakeholders \\*
Dette er også tilnærmelsesvist færdigt, men vi vil gerne have konstruktiv kritik.
\\*

Organisation \\*
Dette afsnit kan springes over.
\\*

Technology \\*
Vi vil gerne have respons på intro-delen og afsnittene som "Google maps" og "Handheld devices". Stor fokus på den metode der er brugt.
\\*

Problem statement\\*
Afsnittet er ikke i sin endelige form, men der ønskes stadigvæk respons.
\\*

Theory \\*
Vi ønsker respons på om teksten er forståeligt.
\\*


Udarbejdet af gruppe b2-6 
\end{document}